\chapter{Konfigurationsdatei}
\begin{tiny}
	SeSc
\end{tiny}

Das System \textbf{OfCourse} benötigt drei Konfigurationsdateien um erfolgreich laufen zu können. In der ersten Datei werden relevante Einstellungen des Systems(Datenbankverbindung, Anzahl der Verbindungen, Überziehungskredit usw) gespeichert. Die zweite Konfigurationsdatei enthält alle nötigen Informationen des E-Mailaccounts(...) über den die Systemnachrichten an die User geschickt werden. Die letzte Datei wird von \textbf{log4j} benötigt, um das Logging Level und den Pfad zum anlegen der Log-Datei zu speichern. Im folgenden wird nun genauer auf alle drei Konfigurationsdateien eingegangen:

\section{Konfiguationsdatei des Systems}
\subsection{Allgemein}

Bei Ersteinrichtung des Systems wird die Datei \textbf{(...).cfg} erstellt und muss vom Systemadministrator bearbeitet werden bevor er den Server startet. Die Datei kann mit einem beliebigen TextEditor geöffnet und editiert werden. Wenn diese Datei im laufendem Serverbetrieb geändert wird, braucht es einen Serverneustart um die Änderungen zu übernehmen.
(Vielleicht noch Risiken beschreiben?)

\subsection{Parameter}

\begin{center}
	\begin{longtable}{|p{4cm} | p{3cm}| p{7cm} | p{2cm} |}
		\hline
		\multicolumn{1}{|c|}{\textbf{Parameter}} & \multicolumn{1}{c|}{\textbf{Name}} & \multicolumn{1}{c|}{\textbf{Details}} & \multicolumn{1}{c|}{\textbf{Eingabetyp}}
		 \\ \hline
		Datenbankname & \textbf{dbname} & Name der Datenbank fuer das System & String \\ \hline
		Datenbankhost & \textbf{dbhost} & Die Adresse zur Datenbank. Angabe als URL & String \\ \hline
		Datenbankport & \textbf{dbport} & Portnummer ueber den die Verbindung zur Datenbank aufgebaut werden kann & Integer \\ \hline
		Datenbankusername & \textbf{dbuser} & Benutzername fuer den Login an der Datenbank & String  \\ \hline
		Datenbankpasswort & \textbf{dbpassword} & Passwort fuer den Login an der Datenbank & String \\ \hline
		Anzahl der Datenbankverbindungen & \textbf{dbconnections} & Legt fest wie viele Verbindungen zur Datenbank zur Verfügung stehen & Integer \\ \hline
		Überziehungskredit & \textbf{bankOverdraft} & Legt den erlaubten Ueberziehungskredit fest & Integer \\ \hline
	\end{longtable}
	

\end{center}
\subsection{Datanbankspezifische Angaben(Wertebereich)}

\begin{itemize}
	\item \emph{dbport:}\\
		Angabe muss zwischen 0 und 65535 liegen.
	\item \emph{dbconnections:}\\
		Angabe muss zwischen 0 und 100 liegen. (Erklaeren warum?)
\end{itemize}

\section{Konfigurationsdatei des E-Mailaccounts}
\subsection{Allgemein}
Bei der Ersteinrichtung der Applikation wird die Konfigurationsdatei \textbf{(...).cfg} erstellt und muss vor dem Serverstart vom Systemadministrator mithilfe eines beliebigen TextEditors bearbeitet und ausgefüllt werden. Wird im Laufendem Serverbetrieb die Datei geändert so muss ein Serverneustart erfolgen um die Änderungen im System zu uebernehmen

\subsection{Parameter}
\begin{center}
	\begin{longtable}{|p{4cm} | p{3cm}| p{7cm} | p{2cm} |}
		\hline
		\multicolumn{1}{|c|}{\textbf{Parameter}} & \multicolumn{1}{c|}{\textbf{Name}} & \multicolumn{1}{c|}{\textbf{Details}} & \multicolumn{1}{c|}{\textbf{Eingabetyp}}
		\\ \hline
		E-Mailadresse & \textbf{mailaddress} & E-Mailadresse mit der das System Nachrichten verschickt & String \\ \hline
		E-Mail SMTP Server & \textbf{smtphost} & Der Postausgangsserver fuer den Nachrichtenversand. Angabe als URL & String \\ \hline
		E-Mail SMTP Port & \textbf{smtpport} & Portnummer des Postausgangsservers & Integer \\ \hline
		E-Mail SMTP Authentifizierung & \textbf{smtpAuth} & SMTP Authentifizierungsart (PLAIN, LOGIN, CRAM-MD5,	DIGEST-MD5 und NTLM) & String  \\ \hline
		E-Mail Username & \textbf{mailusername} & Benutzername fuer den Login am SMTP Server & String \\ \hline
		E-Mail Passwort & \textbf{mailpassword} & Passwort fuer den Login am SMTP Server & String \\ \hline
		
	\end{longtable}
\end{center}
\subsection{E-Mailspezifische Angaben(Wertebereich)}

\begin{itemize}
	\item \emph{smtpport:}\\
	Angabe muss zwischen 0 und 65535 liegen.
	\item \emph{smtpAuth:}\\
	Angabe muss einer existierenden Authentifizierungsart entsprechen. Siehe unten
\end{itemize}

\subsubsection{E-Mailspezifische Angaben(Authentifizierungsarten)}
Die Authentifizierungsart wird von der Serverseite des E-Mail Client zur Verfuegung gestellt. Man unterscheidet hier von:

\begin{itemize}
	\item \emph{PLAIN:}\\
	...
	\item \emph{LOGIN:}\\
	...
	\item \emph{CRAM-MD5:}\\
	...
	\item \emph{SCRAM-SHA-1:}\\
	...
	\item \emph{NTLM:}\\
	...
\end{itemize}

\section{Konfiguartionsdatei von Log4j}	

\subsection{Allgemein}

Nach der Erstinstallation wird die Konfigurationsdatei \textbf{(...).cfg} erstellt und ist mit Default Einstellungen bereits Laufzeitfaehig. Diese Datei kann wiederum vom Systemadministrator mit einem TextEditor angepasst werden. Änderungen im Laufendem Betrieb brauchen wiederum ein Systemneustart um übernommen zu werden.

\subsection{Parameter}

\begin{center}
	\begin{longtable}{|p{4cm} | p{3cm}| p{7cm} | p{2cm} |}
	\hline
	\multicolumn{1}{|c|}{\textbf{Parameter}} & \multicolumn{1}{c|}{\textbf{Name}} & \multicolumn{1}{c|}{\textbf{Details}} & \multicolumn{1}{c|}{\textbf{Eingabetyp}}
	\\ \hline
	Logging Level & loglvl & Setzt das Logging Level ab welcher Stufe gelogt werden soll (Default: All) & String \\ \hline
	Dateipfad & logfilepath & Dateiname und Pfad in der die Logdatei auf dem Server angelegt werden soll & String \\ \hline
	
\end{longtable}

\end{center}

\subsection{Log4jspezifische Angaben(Wertebereich)}

\begin{itemize}
	\item \emph{loglvl:}\\
	Angabe muss einem existierenden Log Level entsprechen. Siehe unten
\end{itemize}

\subsubsection{Log4jspezifische Angaben(Log Levels):}

\begin{itemize}
	\item \emph{FATAL:}\\
	Ein kritischer Fehler ist aufgetreten und das System kann nicht mehr weiter ausgeführt werden.
	\item \emph{ERROR:}\\
	Exceptions wurden geworfen und behandelt. Das System wird alternativ fortgesetzt.
	\item \emph{WARN:}\\
	Auftreten einer unerwarteten Situation. Das System läuft weiter.
	\item \emph{INFO:}\\
	Allgemeine Informationen(z.B. Programm gestartet, Programm beendet, Verbindung zur Datenbank aufgebaut).
	\item \emph{DEBUG:}\\
	allgemeines Debugging(Fehler können leichter gefunden werden).
	\item \emph{TRACE:}\\
	ausführliches Debugging mit Kommentaren.
	\item \emph{ALL:}\\
	Alle Meldungen werden ungefiltert ausgegeben.
	\item \emph{OFF:}\\
	Logging ist deaktiviert.
\end{itemize}


