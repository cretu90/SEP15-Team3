\documentclass[a4paper]{scrreprt}
\usepackage[german]{babel}
\usepackage[german]{translator}
\usepackage[utf8]{inputenc}
\usepackage[T1]{fontenc}
\usepackage{blindtext} 
\usepackage{ae}
\usepackage[bookmarks,bookmarksnumbered]{hyperref}
\usepackage{graphicx}
\usepackage{color}
\usepackage[dvipsnames]{xcolor}

\newcounter{Lc}
\newcounter{Hc}
\newcommand{\stepHc}{\stepcounter{Hc}\setcounter{Lc}{0}}
\newcommand{\resetAllCounter}{\setcounter{Lc}{0}\setcounter{Hc}{1}}

%Glossar
\usepackage[nonumberlist, toc, section=chapter, numberedsection=nolabel]{glossaries}
\makenoidxglossaries

%Glossareinträge
\newglossaryentry{Admin}{name=Administrator, description={Ein Administrator oder kurz Admin ist ein Systemverwalter, welcher besondere (weiterführende) Zugriffsrechte hat. Er betreut und verwaltet das System}}
\newglossaryentry{Trainer}{name=Kursleiter, description={Ein Kursleiter ist ein Trainer, welcher einen oder mehrere Kurse leitet}}
\newglossaryentry{Einheiten}{name=Kurseinheiten, description={Jeder Kurs wird in verschiedene Kurseinheiten unterteilt. Dabei entspricht eine Einheit einem stattfindenden Kurstermin}}
\newglossaryentry{Konfiguration}{name=Konfiguration, description={Eine Konfiguration (meist durch den Systemadministrator) beinhaltet die Installation und die Auswahl von Voreinstellungen sowie beispielsweise das Bearbeiten des Impressums}}
\newglossaryentry{Client}{name=Client, description={Ein Client ist ein Programm, welches auf einem Endgerät ausgeführt wird und Dienstleistungen von einem Server in Anspruch nimmt}}
\newglossaryentry{Server}{name=Server, description={Ein Server ist ein Dienstleister, der in einem Netzwerk Daten oder Ressourcen zur Verfügung stellt}}
\newglossaryentry{JSF}{name=JSF, description={engl. Java Server Faces, ist ein Framework zur Entwicklung von grafischen Benutzeroberflächen für Webprogramme.}}
\newglossaryentry{Tomcat}{name=Apache Tomcat, description={Open Source Applikationsserver zum Ausführen von Web-Anwendungen, welche über Java geschrieben wurden}}
\newglossaryentry{PostgreeSQL}{name=PostgreeSQL, description={PostgreeSQL ist ein freies, objektrelationales Datenbank-Verwaltungssystem}}
\newglossaryentry{SMTP-Server}{name=Kursleiter, description={Dies ist ein Simple Mail Transfer Protokoll welches zum Austausch von E-Mails in Computernetzen dient}}
\newglossaryentry{Orgware}{name=Orgware, description={Die Orgware beschreibt Rahmenbedingungen bei IT-Projekten (also alle organisatorischen, methodischen und personellen Maßnahmen), die weder zu der Software noch zu der Hardware gehören}}
\newglossaryentry{Browser}{name=Browser, description={Browser wie Firefox und Internet Explorer sind spezielle Programme zur Websitedarstellung im World Wide Web}}
\newglossaryentry{URL}{name=URL, description={engl. Uniform Resource Locator, bezeichnet die Adresse einer Webseite}}
\newglossaryentry{Id}{name=Identifikationsnummer, description={ist eine Nummer zur eindeutigen Identifikation}}
\newglossaryentry{Default}{name=Default, description={ist eine voreingestellte Eingabevariable}}
\newglossaryentry{SSL}{name=SSL, description={engl. Secure Sockets Layer ist ein Netzwerkprotokoll zur sicheren Datenübertragung im Internet}}
\newglossaryentry{Dummy}{name=Dummy, description={ ist ein Platzhalter}}
\newglossaryentry{JPG}{name=.jpg, description={ Bezeichnung einer Bilddatei}}
\newglossaryentry{CSS}{name=.css, description={ engl. Cascading Style Sheet, wird für die stylistische Anpassung einer Webseite verwendet}}
\newglossaryentry{String}{name=String, description={Ein String ist eine Folge von vordefinierten Zeichen}}
\newglossaryentry{Boolean}{name=Boolean, description={Ein Boolean ist ein Wahrheitswert, also True oder False}}
\newglossaryentry{Integer}{name=Integer, description={Ein Integer ist ein Datentyp zur Speicherung ganzzahliger Werte}}
\newglossaryentry{float}{name=float, description={Ein Float ist ein Datentyp zur Speicherung von Gleitkommazahlen}}
\newglossaryentry{persistent}{name=persistenten, description={persistent bedeutet längerfristig}}
\newglossaryentry{hashen}{name=gehasht, description={bedeutet verschlüssseln}}
\newglossaryentry{Konsistenz}{name=Konsistenz, description={bedeutet Korrektheit}}




\begin{document}
	\thispagestyle{plain}

\begin{titlepage}
    \begin{center}
\begin{figure}[th]
\centering
\includegraphics[width=0.6\linewidth]{logo/name_blau_ofCourse.jpg}
\end{figure}

    	\begin{title}
        	\title{\Huge{\textbf{Kurseinheiten Manager \\ Pflichtenheft\\}}}

		\end{title}
		\hspace{3cm}

        	Software Engineering Praktikum \\
        	Sommersemester 2015\\
        	Universität Passau\\


        	Betreuer: Andreas Stahlbauer\\
        	\hspace{1,5cm}\\
        	Version: 1.0 \\
        	\hspace{1,5cm}\\
        	Datum: 17.04.2015\\[50pt]
        	\textbf{Team 3} \\
            \ \\
    
        
        
        \begin{tabular}{ | l | l | l | l |}
            \hline
            \textbf{Matrikelnummer} & \textbf{Name} & \textbf{Phase} & \textbf{E-Mail}  \\ \hline
            63097 & Katharina Hölzl & Pflichtenheft & hoelzlka@fim.uni-passau.de \\ \hline
            64504 & Ricky Strohmeier& Entwurf & strohric@fim.uni-passau.de  \\ \hline
            64380 & Martin Bachhuber & Feinspezifikation  & bachhube@fim.uni-passau.de \\ \hline
            64080 & Tobias Fuchs & Implementierung  &  fuchstob@fim.uni-passau.de\\ \hline
            61085 & Sebastian Schwarz & Validierung & sebastian@nrschwarz.de \\ \hline  
            58379 & Patrick Cretu  &  Präsentation & cretu@fim.uni-passau.de \\ \hline
        \end{tabular}
        
        \ \\
        \ \\
        \hspace{3 cm}\\
          \textbf{Arbeitspakete Pflichtenheft} \\
          \ \\
        
        \begin{tabular}{ | l | l |}
        	\hline
        	\textbf{Autor} & \textbf{Verantwortungsbereich} \\ \hline
        	 Katharina Hölzl & Globale Testfälle und Testszenarien, Glossar  \\ \hline
        	 Ricky Strohmeier& Zielbestimmung, Qualitätsbestimmung  \\ \hline
        	 Martin Bachhuber & Produkteinsatz, Produktumgebung, Entwicklungsumgebung  \\ \hline
	         Tobias Fuchs & Produktfunktionen \\ \hline
        	 Sebastian Schwarz & Produktdaten, Produktleistungen \\ \hline  
	         Patrick Cretu  &  Benutzeroberfläche, Ergänzung, Benutzerhilfe  \\ \hline
        \end{tabular}
    \end{center}
\end{titlepage}
 
 


% Platzierung des Inhaltsverzeichnisses
\tableofcontents
 
\chapter{Zielbestimmung}
	\begin{tiny}
		RS
	\end{tiny}
	   Bei der Webanwendung OfCourse handelt es sich um eine Online-Plattform, welche ein allgemeines Planungssystem für Kurse darstellt. Es ermöglicht Kursleitern und Kursteilnehmern die einfache Absprache von Terminen, was bisher zumeist eine große Herausforderung darstellt.
    \section{Musskriterien}      
    	\subsubsection{Allgemeiner Funktionsumfang:}
      		\begin{itemize}
	      		\item Hilfeseite aufrufen \hyperlink{hilfeSeite}{F10-80}
	      		\item Impressum anzeigen \hyperlink{impressum}{F10-70}
	      		\item Allgemeine Geschäftsbedingungen anzeigen \hyperlink{agb}{F10-90}
      		\end{itemize}
     	\subsubsection{Funktionsumfang für anonymen Benutzer:}
       		\begin{itemize}
	      		\item Durchsuchen aller angebotenen Kurse \hyperlink{kursSuche}{F10-50}
	      		\item Kursanzeige sortieren \hyperlink{kursAnzSort}{F10-40}
	       		\item Registrieren \hyperlink{Registrieren}{F10-20}
       		\end{itemize}
     	\subsubsection{Funktionsumfang für alle registrierten Benutzer:}
			\begin{itemize}
				\item Login \hyperlink{login}{F20-10} und Logout \hyperlink{logout}{F20-90}
				\item Kann sich zu Kursen anmelden \hyperlink{kursAnmelden}{F20-120} und abmelden \hyperlink{kursAbmelden}{F20-130}
				\item Kann sich in Kurseinheiten eintragen \hyperlink{kurseinheitAnmelden}{F20-140} und austragen \hyperlink{kurseinheitAbmelden}{F20-150}
				\item Kann eigene Profildaten editieren \hyperlink{profilEdit}{F20-60}
				\item Kann Belegte Kurse anzeigen \hyperlink{kurseAnzeigen}{F20-20}
				\item Einsehen des eigenen Kontostands \hyperlink{kontoAnzeigen}{F20-80} und Geld aufladen \hyperlink{kontoAufladenOn}{F20-180} und \hyperlink{kontoAufladenOff}{F20-190}
				\item Kann Passwort zurücksetzen \hyperlink{passwort}{F20-100}
				\item Funktionsumfang eines Benutzers erweitert sich aufsteigend je nach zugeteilter Rolle
			\end{itemize}
		\subsubsection{Funktionsumfang für Kursleiter:}
			\begin{itemize}
				\item Eigene Kurse bearbeiten \hyperlink{kursEdit}{F30-20}
				\item Kurseinheiten anlegen \hyperlink{kurseinheitAnlegen}{F30-40}, editieren \hyperlink{kurseinheitEdit}{F30-50} und löschen \hyperlink{kurseinheitLoeschen}{F30-80}
				\item Teilnehmer aus Kurs entfernen \hyperlink{kursUserDel}{F30-90}
				\item Nutzer eines Kurses oder einzelner Kurseinheiten benachrichtigen \hyperlink{kursUserMsg}{F30-100}
				\item Accounts bestätigen \hyperlink{addUserKL}{F30-120}
			\end{itemize}
		\subsubsection{Funktionsumfang für Systemadministrator:}
			\begin{itemize}
				\item verwaltet \hyperlink{kursVerwalt}{F40-20}, erstellt \hyperlink{kursErstellen}{F40-10}, bearbeitet \hyperlink{kursBearbeiten}{F40-30} und löscht \hyperlink{kursLoeschen}{F40-40} Kurse und kann zu den Kursen Kursleiter zuordnen \hyperlink{kursKLAdd}{} und wieder entfernen \hyperlink{kursKLDel}{}
				\item Benutzerverwaltung \hyperlink{nutzerVerwalt}{F40-60} einschließlich der Änderung einer Benutzerrolle von Nutzer zu Kursleiter/\gls{Admin} oder \gls{Trainer} zu Nutzer/Administrator \hyperlink{nutzerRolle}{F40-90}
				\item Benutzer suchen \hyperlink{nutzerSuchen}{F40-100}, anlegen \hyperlink{nutzerAnlegen}{F40-50} und löschen \hyperlink{nutzerLoeschen}{F40-80}
				\item Statistiken anzeigen \hyperlink{statistik}{F40-120}
				\item Nutzerguthaben aufladen \hyperlink{guthabenAuf}{F40-130}
				\item System anpassen, beispielsweise eigenes Logo hochladen \hyperlink{logo}{F40-150}, Impressum bearbeiten \hyperlink{impressumBea}{F40-180} oder Registrierungsmodalitäten festlegen \hyperlink{regMod}{F40-160}
			\end{itemize}
			
    \section{Wunschkriterien}
			\begin{itemize}
				\item Englisch als Sprachvariante \hyperlink{spracheAendern}{F10-100}
				\item Terminplaner \hyperlink{terminplaner}{F20-110}
	     		\item Einschränken des angezeigten Kursangebots \hyperlink{kursEinschraenken}{F10-110}
			\end{itemize}
			
	\section{Abgrenzungskriterien}
     		\begin{itemize}
	     		\item Keine Raumplanung/-verwaltung für die Kurse
	     		\item Keine Einsicht in fremde Nutzerprofile, nur Admin kann Nutzerdaten einsehen
	     		\item Kein systeminternes Kommunikationssystem zwischen den Benutzern
	     		\item Keine Funktionen zur erleichterten Bedienung für Nutzer mit Behinderung
     		\end{itemize}
        
  
\chapter{Produkteinsatz}
    \section{Anwendungsbereiche}
		 Der \gls{Einheiten}-Manager dient der Organisation und Planung von Veranstaltungen. Benutzer können sich im System über verschiedene Veranstaltungen informieren.   
     
	\section{Zielgruppen}
		 \begin{enumerate}
		 	\item Betreiber: Die Webanwendung soll von Betreibern unterschiedlichsten Fachbereichs verwendet werden können, beispielsweise Betreiber von Sportvereinen oder Lehrinstituten, welche ihre angebotenen Veranstaltungen und Kurse per Internet organisieren möchten.
		 	\item Benutzer: Es wird zwischen vier verschiedenen Benutzern unterschieden:
		 	\begin{itemize}
		 		\item Anonymer Benutzer: Anonyme Benutzer benötigen rudimentäre Kenntnisse im Umgang mit dem Computer, um nach 	Veranstaltungen suchen und informieren zu können.
		 		\item Registrierter Benutzer: Registrierte Benutzer benötigen ebenso wie anonym Benutzer keine besonderen Vorkenntnisse. Sie benötigen lediglich eine E-Mail-Adresse, um sich im System registrieren zu können und somit ihr Benutzerkonto erstellen. Sie können nach Veranstaltungen suchen und sich für solche an- oder abmelden.
		 		\item Kursleiter: Ein Kursleiter benötigt gleichfalls wenig besondere IT-Kenntnisse. Er sollte allerdings über alle nötigen Informationen seiner Veranstaltung in Kenntnis sein, um jene anlegen und verwalten zu können.
		 		\item Administrator: Ein Administrator ist für die Verwaltung und \gls{Konfiguration} des gesamten Systems zuständig. Aufgrund dessen werden grundlegende IT-Kenntnisse vorausgesetzt.
		 	\end{itemize}	
		 \end{enumerate}  
        
    
	\section{Betriebsbedingungen}
	       \begin{itemize}
		       	\item Betriebszeit/Betriebsdauer: Ununterbrochener Betrieb mit Ausnahme einer Wartungszeit von 2 Stunden in der Woche bzw. 8 Stunden im Monat, welche aber nicht zur Hauptnutzungszeit des Systems abgehalten wird, sofern dies möglich ist.
		       	\item  Im laufenden Betrieb ist keine Aufsicht vonnöten.
	       \end{itemize}
	
			
 
\chapter{Produktumgebung}
	\section{Software}
        \begin{itemize}
      		\item \gls{Client}:
	      		\begin{itemize}
	      			\item Mozilla Firefox 37
	      			\item Internet Explorer 11
	      			\item Google Chrome 42
	      		\end{itemize}
          	\item \gls{Server}:
	            \begin{itemize}
	            	\item Java 8
	            	\item \gls{JSF} 2.2 (Entwicklungsframework)
	            	\item \gls{Tomcat} 8 (Webserver)
	            	\item \gls{PostgreeSQL} 9 (Datenbank)  
	            	\item \gls{SMTP-Server}
	            \end{itemize}
        \end{itemize}
        
    \section{Hardware}   
        \begin{itemize}
          	\item Client:
	            \begin{itemize}
	            	\item Endgerät mit Internetzugang
	            	\item Mindestauflösung: 480 x 800 Pixel
	            \end{itemize}
          	\item Server:
	           \begin{itemize}
		           	\item Rechner mit Internetzugang
		           	\item Rechner mit mindestens 5 GB Festplattenspeicher für Datenbank und Software
		           	\item Intel(R) Core(TM) i5-2400 CPU, 8GB RAM 
	           \end{itemize}
        \end{itemize}
        
     \section{Orgware}
             Es wird keine \gls{Orgware} benötigt.

\resetAllCounter
\newcommand{\Func}[1]{\stepcounter{Lc}\textcolor{Blue}{\textbf{/F\arabic{Hc}0-\arabic{Lc}0/} #1} \\}
\newcommand{\FuncW}[1]{\stepcounter{Lc}\textcolor{Green}{\textbf{/F\arabic{Hc}0-\arabic{Lc}0W/} #1} \\}
\newcommand{\FuncBlue}[1]{\textcolor{Blue}{\textbf{#1}}}
\newcommand{\FuncGreen}[1]{\textcolor{Green}{\textbf{#1}}}

\chapter{Produktfunktionen}
Die Produktfunktionen sind nach Benutzergruppen geordnet. Wunschkriterien sind mit nachgestelltem 'W' gekennzeichnet.
	\section{Anonymer Benutzer}
		\subsection{Grundfunktionen}
			\begin{itemize}
				\item \Func{Aufruf der Startseite.}
					Die Startseite kann mit jedem unterstützten \gls{Browser} (vgl. Produktumgebung -> 3.1 Software -> Client) via \gls{URL} aufgerufen werden.
				\item \Func{Registrierung im System durch Ausfüllen eines Formulars.} \hypertarget{Registrieren}
					Der Nutzer kann sich durch Ausfüllen eines Registrierungsformulars im System registrieren. Die Anwendung generiert für den neu registrierten Nutzer automatisch eine im System eindeutige Identifikationsnummer. Nach Abschicken des Formulars wird an die angegebene E-Mailadresse eine Bestätigungsmail  mit einem Verifizierungslink geschickt, welcher dazu dient, die E-Mailadresse des neuen Benutzers zu bestätigen. Der Nutzer wird dann auf die 'Anmelden' - Seite des Systems weitergeleitet. Ob sich der Benutzer nun schon in das System einloggen kann, hängt von der Art der Accountaktivierung ab. Diese wird durch den Administrator festgelegt(\FuncBlue{/F40-160/}). Der Account ist entweder durch die E-Mailverifikation automatisch aktiviert worden oder muss manuell durch einen Kursleiter oder einen Administrator aktiviert werden(\FuncBlue{/F30-120/}).
				\item \Func{Anzeigen des Kursangebots.} \hypertarget{kursAngAnz}
					Durch Klicken der Schaltfläche 'Kursangebot' wird dem Nutzer das gesamte Kursangebot angezeigt. Die Anzeige ist automatisch aufsteigend nach dem 'Beginn' der nächsten Kurseinheit des Kurses sortiert. 
				\item \Func{Sortieren des angezeigten Kursangebots.} \hypertarget{kursAnzSort}
					Die angezeigten Kurse können entweder nach 'Titel' des Kurses alphabetisch oder nach dem 'Beginn' der nächsten Kurseinheit des Kurses chronologisch sortiert werden.
				\item \Func{Suche nach Kursen.} \hypertarget{kursSuche}
					Das Kursangebot kann durch Eingabe des Kurstitels und Betätigen der 'Suchen' - Schaltfläche nach einem bestimmten Kurs durchsucht werden.
				\item \Func{Anzeigen der Kursdetails} 
				    Durch Klicken auf einen Kurs des Kursangebots wird der anonyme Nutzer zur 'Kursdetails' - Seite des Kurses weitergeleitet. Auf der Seite werden die Identifikationsnummer des Kurses, der 'Titel', die 'Beschreibung', die 'Trainer' und die 'Kurseinheiten' des Kurses angezeigt(\FuncBlue{/D20-10/}). Eine Anmeldung zum Kurs oder zu einer Kurseinheit und die Anzeige der Kursteilnehmer ist für den anonymen Nutzer nicht möglich.
				\item \Func{Anzeige des Impressums.}\hypertarget{impressum}
					Das Impressum kann von jeder Seite des Systems aus angezeigt werden.
				\item \Func{Anzeige der Hilfeseite.} \hypertarget{hilfeSeite}
					Von jeder Seite des Systems kann die Hilfeseite zur entsprechenden Seite des Systems oder die komplette Hilfeseite aufgerufen werden.
				\item \Func{Anzeige der Allgemeinen Geschäftsbedingungen.} \hypertarget{agb}
					Die Allgemeinen Geschäftsbedingungen können von jeder Seite des Systems aus angezeigt werden.
				\item \FuncW{Wechsel der Anzeigesprache.} \hypertarget{spracheAendern}
					Von jeder Seite des Systems aus kann durch Klicken einer entsprechenden Schaltfläche die Anzeigesprache gewechselt werden. Als Default ist 'Deutsch' ausgewählt. Als weitere Sprache ist 'Englisch' vorgesehen.
				\item \FuncW{Einschränken des angezeigten Kursangebots.} \hypertarget{kursEinschraenken}
					Der Nutzer hat die Möglichkeit das angezeigte Kursangebot auf verschiedene Zeiträume einzuschränken. Vorgesehene Anzeigezeiträume sind 'Tagesangebot' und 'Wochenangebot'. Die Auswahl des Anzeigezeitraums ist über eine Drop-Down-Liste möglich. Bestätigt wird die Auswahl durch Klicken der 'Anzeigen' - Schaltfläche. 
			\end{itemize}

	\section{Registrierter Benutzer}
	Der registrierte Benutzer verfügt über sämtliche Funktionen wie der anonyme Benutzer. Zusätzlich kann er auf folgende Funktionen zurückgreifen.
		\stepHc
		\subsection{Grundfunktionen}
			\begin{itemize}
				\item \Func{Sicheres Einloggen ins System.} \hypertarget{login}
					Durch Eingabe des Benutzernamen, des Passworts und Klicken der 'Anmelden' - Schaltfläche kann sich der Benutzer ins System einloggen. Die Anmeldedaten werden dabei mittels \gls{SSL} übertragen. Nach erfolgreichem Anmelden wird der Nutzer auf die 'Meine Kurse' - Seite weitergeleitet.
				\item \Func{Anzeigen der angemeldeten Kurse.} \hypertarget{kurseAnzeigen}
					Die Kurse zu denen der registrierte Benutzer angemeldet ist werden gesammelt auf der 'Meine Kurse' - Seite angezeigt.
				\item \Func{Detailanzeige der angemeldeten Kurse.}
					Durch Klicken auf einen angemeldeten Kurs werden die Details dieses Kurses angezeigt. Der registrierte Nutzer bekommt die Daten aus \FuncBlue{/F10-60/} angezeigt. Zusätzlich kann der registrierte Nutzer in seinem angemeldeten Kurs die Teilnehmerliste einsehen und bekommt Details der Kurseinheiten, wie etwa 'Ort' und 'Preis'(vgl. \FuncBlue{/D30-10/}), angezeigt. Außerdem hat der registrierte Benutzer die Möglichkeit sich zu Kurseinheiten an-/abzumelden und sich aus dem Kurs auszutragen.
				\item \Func{Anzeige der Teilnehmer eines angemeldeten Kurses}
				    Durch Betätigen der 'Zu den Teilnehmern' - Schaltfläche auf der 'Kursdetails' - Seite eines angemeldeten Kurses kann der registrierte Nutzer die Teilnehmerliste des Kurses einsehen. 
				\item \Func{Anzeigen der Profilseite.}
					Der registrierte Benutzer besitzt eine Profilseite auf welcher seine Daten gesammelt angezeigt werden.
				\item \Func{Ändern der eigenen Benutzerdaten.} \hypertarget{profilEdit}
					Der registrierte Benutzer kann durch Klicken der Schaltfläche 'Bearbeiten' seine eingegebenen Daten bearbeiten. Sollte die E-Mailadresse geändert werden, wird an die neue E-Mailadresse wie beim Registrierungsprozess eine Bestätigungsmail mit einem neuen Verifizierungslink geschickt, um die neue E-Mail-Adresse auf Korrektheit zu überprüfen. Der registrierte Nutzer kann alle seine Benutzerdaten ändern mit Ausnahme seines Kontostands, seiner Benutzerrolle und der vom System generierten Identifikationsnummer. Durch Betätigen der 'Speichern' - Schaltfläche werden die Daten gespeichert.
				\item \Func{Hochladen eines Profilbilds.}
					Der registrierte Benutzer kann ein Profilbild hochladen. Das Bild muss das .jpg - Format haben. Anforderungen bezüglich Auflösung und Größe des Bildes werden während der Produktentwicklung festgelegt. Als Default ist ein \gls{Dummy} - Bild gesetzt.
				\item \Func{Anzeigen des aktuellen Kontostands.} \hypertarget{kontoAnzeigen}
					Der registrierte Benutzer kann den aktuellen Kontostand seines Guthabenkontos auf seiner Profilseite einsehen. 
				\item \Func{Aus dem System abmelden mittels 'Abmelden' - Schaltfläche.} \hypertarget{logout}
					Der registrierte Benutzer kann sich  von jeder Seite des Systems aus durch Klicken der Schaltfläche 'Abmelden' aus dem System abmelden und wird auf die Anmeldeseite weitergeleitet.
				\item \Func{'Passwort vergessen' - Schaltfläche.} \hypertarget{passwort}
					Klickt der Benutzer die 'Passwort vergessen' - Schaltfläche, so wird er aufgefordert seine im System hinterlegte E-Mailadresse einzugeben. An diese wird dann ein automatisch generiertes Passwort geschickt.
			\end{itemize}	

		\subsection{Kursfunktionen}
			\begin{itemize}
				\item \Func{Detailansicht der Kurse.}
					Der registrierte Nutzer kann sich eine Detailansicht der angebotenen Kurse anzeigen lassen. Angezeigt werden die Daten aus \FuncBlue{/F10-60/}. Außerdem kann sich der Nutzer in der Detailansicht zum Kurs anmelden. Eine Einsicht in die Teilnehmerliste des Kurses ist erst nach Anmeldung zum Kurs möglich.
				\item \Func{Zu Kurs anmelden.} \hypertarget{kursAnmelden}
					Der registrierte Nutzer kann sich zu beliebig vielen Kursen anmelden. 
					Dafür klickt er in der Detailansicht des gewünschten Kurses auf die Schaltfläche 'Anmelden'.
					Bei der Anmeldung hat der registrierte Nutzer die Möglichkeit sich für Benachrichtigungen zu diesem Kurs zu registrieren(vgl. \FuncBlue{/F20-170/}).
				\item \Func{Von Kurs abmelden.}  \hypertarget{kursAbmelden}
					Der registrierte Nutzer kann sich von Kursen abmelden zu denen er angemeldet ist. Dafür klickt er in der Detailansicht des gewünschten Kurses auf die Schaltfläche 'Abmelden'. Meldet sich der registrierte Benutzer von einem Kurs ab, meldet er sich damit auch automatisch von allen Kurseinheiten ab.
				\item \Func{Zu Kurseinheit anmelden.} \hypertarget{kurseinheitAnmelden}
					Um sich zu Kurseinheiten eines Kurses anmelden zu können, muss der Nutzer in die Detailansicht des Kurses wechseln. Dort findet er die zu diesem Kurs verfügbaren Einheiten und deren Status. Ist eine Kurseinheit noch nicht voll und ist der Nutzer noch nicht zu dieser Einheit angemeldet, kann er sich anmelden. Der Nutzer hat außerdem die Möglichkeit durch Klicken der Schaltfläche 'Alle auswählen' und anschließender Betätigung der Schaltfläche 'Speichern' sich zu allen Einheiten des Kurs auf einmal anzumelden. Vor Betätigen der Schaltfläche 'Speichern' hat der Nutzer aber auch noch die Möglichkeit einzelne Einheiten von der Anmeldung auszuschließen. Voraussetzung um sich zu kostenpflichtigen Kurseinheiten anmelden zu können ist ein ausreichendes Guthaben auf dem Konto bzw. keine unzulässige Überziehung des Guthabenkontos.
				\item \Func{Von Kurseinheit abmelden.} \hypertarget{kurseinheitAbmelden}
					Der registrierte Benutzer kann sich von Kurseinheiten abmelden, zu denen er angemeldet ist. Um sich von Kurseinheiten eines Kurses abmelden zu können, muss der Nutzer in die Detailansicht des Kurses wechseln. Dort findet er die zu diesem Kurs verfügbaren Einheiten.
					Er kann nun die Einheiten auswählen von denen er sich abmelden will und durch Betätigen der Schaltfläche 'Speichern' meldet er sich von den gewählten Einheiten ab. Der für kostenpflichtige Einheiten bezahlte Betrag wird automatisch auf das Konto des registrierten Benutzers zurückgebucht.
					Eine Abmeldung von einer Kurseinheit ist nur bis drei Stunden vor deren Beginn möglich. Sollte nämlich aufgrund der Abmeldung des registrierten Nutzers die minimale Teilnehmerzahl der Kurseinheit unterschritten werden, kann der Kursleiter entscheiden, ob die Kurseinheit stattfindet oder nicht. Durch die Begrenzung der Abmeldefrist hat der Kursleiter somit noch eine gewisse Reaktionszeit und kann gegebenenfalls die anderen Teilnehmer über den Ausfall der Kurseinheit informieren. 
				\item \Func{Nach Kursen suchen - erweitert.}
					Der registrierte Nutzer hat die Möglichkeit das gesamte Kursangebot zu durchsuchen. Über eine Drop-Down-Liste kann der Nutzer einen Suchparameter auswählen. Nach Eingabe eines Suchbegriffs und Klicken der Schaltfläche 'Suchen' wird das Kursangebot nach dem Suchbegriff bzgl. des Suchparameters durchsucht und das Ergebnis angezeigt. 
					Als Suchparameter sind vorgesehen 'ID', 'Titel', 'Kursleiter', 'Beginn', 'Kostenpflichtig', 'Kostenlos'.
				\item \Func{Für Kursbenachrichtigungen eintragen}
					Der registrierte Nutzer hat die Möglichkeit sich bei der Anmeldung zu einem Kurs durch Auswahl einer Checkbox in der Detailansicht des Kurses für Benachrichtigungen zu diesem Kurs zu registrieren oder nicht. Per Default ist die Checkbox nicht ausgewählt.
			\end{itemize}

		\subsection{Bezahlfunktionen}
			\begin{itemize}
				\item \Func{Guthabenkonto aufladen - online} \hypertarget{kontoAufladenOn}
					Der registrierte Nutzer besitzt ein Guthabenkonto. Dieses kann er online mittels Kreditkarte aufladen. Dafür betätigt Benutzer die Schaltfläche 'Konto aufladen' auf seiner Profilseite. Er wird auf die 'Konto aufladen' - Seite weitergeleitet. Dort kann er seine Kreditkartendaten und den aufzuladenden Betrag eingeben. Die Kreditkartenabwicklung erfolgt dabei über die Infosun-Bank. Die Übertragung der Kreditkartendaten erfolgt mittels SSL. Der aufgeladene Betrag ist sofort auf dem Guthabenkonto verfügbar und kann auch sofort verwendet werden. Aus Sicherheitsgründen werden keine Kreditkartendaten im System gespeichert.
				\item \Func{Guthabenkonto aufladen - offline} \hypertarget{kontoAufladenOff}
					Der registrierte Nutzer kann sein Guthabenkonto auch offline aufladen. Dies ist mittels Überweisung oder Barzahlung an den Betreiber oder einen Kursleiter möglich. 
					Der Nutzer überweist dafür den Betrag unter Angabe seiner Identifikationsnummer und seines Namens  auf das Konto des Betreibers oder bezahlt bar. Der bezahlte Betrag ist erst auf dem Guthabenkonto des registrierten Benutzers vorhanden, wenn er vom Betreiber oder einem der Kursleiter gutgeschrieben wurde.
				\item \Func{Kurseinheiten bezahlen}
					Die Bezahlung von Kurseinheiten wird automatisch bei der Anmeldung zur Kurseinheit erledigt. Der registrierte Nutzer kann sich also auch nur zu der Kurseinheit anmelden, wenn sein Guthaben auf dem Konto für den Preis der Einheit ausreicht bzw. sein Kontostand nicht den vom Betreiber festgelegten Überziehungskredit überschreitet.
				\item \Func{Rückbuchung des Kurspreises bei Abmeldung oder Kursausfall}
					Meldet sich der registrierte Nutzer von einer Kurseinheit ab oder fällt die Kurseinheit aufgrund zu geringer Teilnehmerzahl oder wegen eines anderen Grunds aus, so wird der Kaufpreis der Einheit automatisch auf das Konto des registrierten Nutzers zurückgebucht.
			\end{itemize}	
			
		\subsection{Terminplaner}
			\begin{itemize}
				\item \FuncW{Terminplaner.} \hypertarget{terminplaner}
					Der registrierte Benutzer hat einen persönlichen Terminplaner. Dieser wird in der Wochenansicht dargestellt. Der Terminplaner unterstützt nur die Anzeige von einstündigen Slots. Ist ein Termin kürzer, wird dennoch der Slot für die ganze Stunde belegt. Ist der Termin länger als eine Stunde werden dementsprechend mehrere einstündige Slots belegt.
					Außerdem ist darauf zu achten, dass die Slots jeweils zur vollen Stunde beginnen und zur nächsten vollen Stunde enden. (Analog dem Prinzip eines Stundenplans)
				\item \FuncW{Anzeigen des Terminplaners.}
					Der persönliche Terminplaner des registrierten Benutzers kann angezeigt werden. 
				\item \FuncW{Automatische Eintragung von angemeldeten Kurseinheiten.}
					Die angemeldeten Kurseinheiten des registrierten Benutzers werden automatisch in dessen Terminplaner eingetragen.
				\item \FuncW{Automatische Entfernung von abgemeldeten Kurseinheiten.}	
					Meldet sich der registrierte Benutzer aus einer Kurseinheit ab oder fällt eine Kurseinheit aus, so wird diese aus dem Terminplaner gelöscht.
				\end{itemize}   

	\section{Kursleiter}
	Jedem Kursleiter stehen auch alle Funktionen eines registrierten Benutzers zur Verfügung. Darüber hinaus kann ein Kursleiter auf folgenden Funktionen zurückgreifen.
	\stepHc
		\subsection{Kurse}
			\begin{itemize}
				\item \Func{Eigene Kurse anzeigen.}
					Auf der Kursleiterseite werden die Kurse, welche der Kursleiter trainiert, gesammelt angezeigt.
				\item \Func{Eigenen Kurs editieren.} \hypertarget{kursEdit}
					Der Kursleiter hat die Möglichkeit, die Kurse, die er  trainiert zu editieren. Dafür klickt er in seinen Trainingskursen den gewünschten Kurs an und wird auf die 'Kursdetails' - Seite weitergeleitet. Durch Betätigen der 'Bearbeiten' - Schaltfläche kann er die Kursdaten bearbeiten. Er kann alle angezeigten Kursdaten verändern mit Ausnahme der vom System generierten Identifikationsnummer. Durch Betätigen der 'Speichern' - Schaltfläche werden die Kursdaten gespeichert.
				\item \Func{Eigenen Kurs löschen.}
					Der Kursleiter hat die Möglichkeit, die Kurse, die er  trainiert zu löschen. Dafür klickt er in seinen Trainingskursen den gewünschten Kurs an und wird dann auf die 'Kursdetails' - Seite weitergeleitet. Dort kann er den Kurs durch Betätigen der 'Kurs löschen' - Schaltfläche löschen.
				\item \Func{Kurseinheit anlegen.} \hypertarget{kurseinheitAnlegen}
					Der Kursleiter hat die Möglichkeit Kurseinheiten für einen Kurs anzulegen, welchen er trainiert. Dafür wechselt er in die 'Kursdetails' - Oberfläche. Dort betätigt er die Schaltfläche 'Kurseinheit anlegen'. Er wird auf eine Seite weitergeleitet, auf welcher er die Daten der Kurseinheit eintragen kann. Das System generiert für die Kurseinheit eine im System eindeutige Identifikationsnummer. Durch Betätigen der 'Speichern' - Schaltfläche wird die Kurseinheit angelegt. Die Kurseinheiten können nur in dem Zeitraum zwischen Startzeitpunkt und Endzeitpunkt des zugehörigen Kurses stattfinden.
				\item \Func{Kurseinheit editieren.} \hypertarget{kurseinheitEdit}
					Der Kursleiter hat die Möglichkeit Kurseinheiten der eigenen Kurse zu editieren. Dafür klickt er in der Detailansicht des Kurses die 'Bearbeiten' - Schaltfläche der Kurseinheit an und wird auf die 'Kurseinheit bearbeiten' - Oberfläche weitergeleitet. Hier kann der Kursleiter die Daten der Kurseinheit mit Ausnahme der vom System generierten Identifikationsnummer bearbeiten. Außerdem kann der Kursleiter hier manuell durch Eingabe der Daten einen Teilnehmer zur Kurseinheit hinzufügen.
				\item \Func{Teilnehmer zu Kurseinheit hinzufügen.}
					Der Kursleiter hat die Möglichkeit durch Eingabe der Daten eines Benutzers diesen als Teilnehmer zu einer einzelnen Kurseinheit einer seiner Kurse einzutragen. Dafür muss der Kursleiter auf die 'Kurseinheit bearbeiten' - Seite der entsprechenden Kurseinheit wechseln.
				\item \Func{Teilnehmer aus Kurseinheit entfernen.}
				    Der Kursleiter hat die Möglichkeit Teilnehmer aus einer Kurseinheit einer seiner Kurse zu entfernen. Dafür muss der Kursleiter auf die 'Kurseinheit bearbeiten' - Seite der entsprechenden Kurseinheit wechseln. Dort wählt er den zu löschenden Teilnehmer aus der Teilnehmerliste der Kurseinheit aus und betätigt die 'Löschen' - Schaltfläche. Es besteht auch die Möglichkeit mehrere Teilnehmer gleichzeitig zu löschen.
				\item \Func{Kurseinheit löschen.} \hypertarget{kurseinheitLoeschen}
					Der Kursleiter kann Kurseinheiten eigener Kurse löschen. Dafür wechselt er durch Betätigen der 'Bearbeiten' - Schaltfläche der zu löschenden Kurseinheit auf die 'Kurseinheit bearbeiten' - Oberfläche. Durch Betätigen der 'Löschen' - Schaltfläche kann er die Kurseinheit löschen.
				\item \Func{Teilnehmer aus Kurs entfernen.} \hypertarget{kursUserDel}
					Der Kursleiter kann einzelne Teilnehmer aus einem seiner eigenen Kurse entfernen. Dafür wechselt er über 'Kursdetails' - Oberfläche auf die Teilnehmerliste des Kurses. Dort wählt er den zu entfernenden Teilnehmer aus und entfernt diesen durch Betätigen der 'Löschen' - Schaltfläche aus dem Kurs. Es besteht auch die Möglichkeit mehrere Teilnehmer gleichzeitig zu entfernen.
			\end{itemize}

		\subsection{Benachrichtigungen}
			\begin{itemize}
				\item \Func{Benachrichtigung an Benutzergruppe senden.} \hypertarget{kursUserMsg}
					Der Kursleiter kann Benachrichtigungen an alle Teilnehmer seines Kurses oder nur an die Teilnehmer einer Kurseinheit seines Kurses senden.
				\item \Func{Benachrichtigung bei Kursänderung senden.}	
					Nimmt der Kursleiter Änderungen an einem seiner Kurse oder seiner Kurseinheiten vor, so kann er entscheiden, ob er alle Kursteilnehmer, nur die Teilnehmer der betreffenden Kurseinheit oder keinen Teilnehmer benachrichtigt.
			\end{itemize}

		\subsection{Verwaltung}
			\begin{itemize}
				\item \Func{Account aktivieren} \hypertarget{addUserKL}
					Der Kursleiter hat die Möglichkeit Accounts zu aktivieren, falls dies durch den Administrator in den Einstellungen für die Accountaktivierung so festgelegt wurde. Ist dies der Fall so wird der Kursleiter durch Betätigen der 'Benutzer aktivieren' - Schaltfläche auf die 'Benutzer aktivieren' - Seite weitergeleitet. Dort findet er eine Auflistung aller Benutzer, die ihre E-Mailadresse bereits verifiziert haben, aber deren Account noch nicht aktiviert wurde. Der Kursleiter kann nun den Benutzer auswählen, dessen Account er aktivieren will. Durch Betätigen der 'Benutzer aktivieren' - Schaltfläche wird der Account des Benutzers aktiviert und dieser kann sich nun im System einloggen. 
					Es besteht auch die Möglichkeit mehrere Accounts gleichzeitig zu aktivieren. Nach Aktivierung der Benutzeraccounts erhalten die entsprechenden Nutzer eine automatische E-Mail, dass sie sich ab sofort im System anmelden können.
			\end{itemize}

	\section{Systemadministratoren}
	Jedem Systemadministrator stehen auch alle Funktionen eines Kursleiters zur Verfügung. Darüber hinaus kann ein Systemadministrator auf folgende Funktionen zurückgreifen.
	\stepHc
		\subsection{Kurse}
			\begin{itemize}
				\item \Func{Kurs anlegen.} \hypertarget{kursErstellen}
					Der Systemadministrator kann einen neuen Kurs anlegen. Dafür betätigt der Administrator die entsprechende Schaltfläche auf der Administratorseite und wird auf die 'Kurs anlegen' - Seite weitergeleitet. Die Anwendung generiert automatisch eine für den Kurs im System eindeutige Identifikationsnummer. Der Administrator hat die Möglichkeit die Daten des Kurses, wie etwa eine Beschreibung oder den Veranstaltungsort einzugeben. Notwendige Eingaben sind der Startzeitpunkt, der Endzeitpunkt und mindestens ein Kursleiter für den Kurs. Innerhalb des Zeitraums von Startzeitpunkt bis Endzeitpunkt können Kurseinheiten stattfinden.
				\item \Func{Kurse verwalten.} \hypertarget{kursVerwalt}
					Der Systemadministrator kann sich alle Kurse anzeigen lassen. Dafür betätigt er die Schaltfläche 'Kurse verwalten' und wird auf die 'Kursangebot durchsuchen' - Oberfläche weitergeleitet. Dort kann er sich das gesamte Kursangebot anzeigen lassen oder nach bestimmten Kursen bezüglich verschiedener Suchparameter suchen. Durch Betätigen der Schaltfläche 'Anzeigen' werden die gesuchten Kurse angezeigt. Durch Klicken auf einen Kurs wird der Administrator auf die 'Kursdetails' - Seite weitergeleitet.
				\item \Func{Kurs bearbeiten.} \hypertarget{kursBearbeiten}
					Der Systemadministrator hat die Möglichkeit einen beliebigen Kurs zu bearbeiten. Dafür wechselt er in die Detailansicht des zu editierenden Kurses. Durch Betätigen der 'Bearbeiten' - Schaltfläche  kann der Administrator nun die angezeigten Kursdaten editieren(vgl. \FuncBlue{/D20-10/}). Einzige Ausnahme ist die vom System generierte Identifikationsnummer des Kurses. Des Weiteren kann er Kurseinheiten anlegen, löschen oder editieren, Kursleiter zum Kurs hinzufügen oder entfernen und die Teilnehmer des Kurses editieren.
				\item \Func{Kurs löschen.} \hypertarget{kursLoeschen}
					Der Systemadministrator hat die Möglichkeit einen beliebigen Kurs zu löschen. Dafür wechselt er auf die 'Kursdetails' - Seite des zu löschenden Kurses. Durch Betätigen der 'Kurs löschen' - Schaltfläche wird der Kurs und damit auch alle zugehörigen Kurseinheiten gelöscht.
			\end{itemize}

	\subsection{Benutzer}
			\begin{itemize}
				\item \Func{Neuen Benutzer anlegen.} \hypertarget{nutzerAnlegen}
					Der Systemadministrator hat die Möglichkeit einen neuen Benutzer anzulegen. Dafür betätigt er die Schaltfläche 'Benutzer anlegen' auf der Administratorseite und wird auf eine Seite 'Benutzer anlegen' weitergeleitet. 
					Die Anwendung generiert automatisch eine im System eindeutige Identifikationsnummer für den Benutzer. Der Systemadministrator kann die Benutzerdaten eingeben, ein Initialpasswort setzen und die Rolle des Benutzers festlegen. Die Benutzerrolle wird mittels einer Drop-Down - Liste aus den Möglichkeiten 'Kursleiter', 'Administrator','Benutzer' ausgewählt. Durch Betätigen der Schaltfläche 'Speichern' wird der neue Nutzer angelegt. Der vom Administrator erstellte Benutzer kann sich sofort ins System einloggen und muss nicht mehr aktiviert werden.
				\item \Func{Benutzer verwalten.} \hypertarget{nutzerVerwalt}
					Der Systemadministrator besitzt eine Übersicht über alle Benutzer. Diese erreicht er über die Schaltfläche 'Benutzer suchen' der Administratorseite. Auf der 'Benutzer durchsuchen' - Seite hat der Administrator die Möglichkeit sich alle Benutzer anzeigen zu lassen oder nach Benutzern zu suchen.  Dafür wählt er aus einer Drop - Down - Liste einen Parameter aus und gibt in ein Eingabefeld seinen Suchbegriff ein. Durch Betätigen der Schaltfläche 'Anzeigen' werden die gesuchten Benutzer angezeigt. Durch Klicken auf einen Benutzer wird der Administrator auf die 'Profil' - Seite des Benutzers weitergeleitet.
				\item \Func{Benutzer bearbeiten.} \hypertarget{nutzerBearbeiten}
					Der Systemadministrator hat die Möglichkeit einen beliebigen registrierten Nutzer zu bearbeiten. Dafür wechselt er auf die 'Profil' - Seite des zu bearbeitenden Nutzers. Durch Betätigen der 'Bearbeiten' - Schaltfläche kann er Administrator nun die Benutzerdaten editieren oder die Benutzerrolle verändern. Der Administrator kann alle Benutzerdaten verändern mit Ausnahme der vom System generierten Identifikationsnummer des Benutzers. Durch Betätigen der 'Speichern' - Schaltfläche werden die Daten gespeichert.
				\item \Func{Benutzer löschen.} \hypertarget{nutzerLoeschen}
					Der Systemadministrator hat die Möglichkeit einen beliebigen Benutzer zu löschen. Dafür wechselt er auf die 'Profil' - Seite des zu löschenden Benutzers. Durch Betätigen der 'Benutzer löschen' - Schaltfläche wird der Benutzer gelöscht. Nur der Administrator hat die Möglichkeit Benutzer zu löschen. Andernfalls könnte ein \glqq Nicht - Administrator\grqq \ einen Benutzer und damit auch sein Guthaben bzw. seine eventuellen Schulden löschen.
				\item \Func{Benutzerrolle verändern.} \hypertarget{nutzerRolle}
					Der Systemadministrator hat die Möglichkeit die Rolle jedes Nutzers zu verändern. Als Benutzerrollen sind 'Benutzer', 'Kursleiter' und 'Administrator' vorgesehen. Die Benutzerrolle kann auf der 'Benutzer bearbeiten' - Seite verändert werden.
				\item \Func{Benutzer suchen.} \hypertarget{nutzerSuchen}
					Der Systemadministrator hat die Möglichkeit nach Benutzern zu suchen. Dafür wählt er auf er 'Administrator' - Oberfläche die entsprechende Schaltfläche aus. Er wird auf die 'Benutzer durchsuchen' - Seite weitergeleitet, wo er die entsprechenden Suchparameter auswählen und seinen Suchbegriff eingeben kann. Durch Betätigen der Schaltfläche 'Anzeigen' wird das Suchergebnis angezeigt.
				\item \Func{Account aktivieren.}
				    Der Administrator kann genauso wie der Kursleiter Accounts aktivieren(vgl.\FuncBlue{/F30-120/}), falls dies vom Administrator so eingestellt wurde. Der Administrator wählt dafür die 'Benutzer verwalten' - Schaltfläche auf der Administratorseite aus und wird auf die 'Benutzer durchsuchen' - Seite weitergeleitet. Dort wählt er aus der Drop-Down-Liste die Option 'Nicht aktiviert' aus. Die nicht aktivierten Benutzer werden nun angezeigt. Außerdem werden nun auf der Seite zusätzlich die entsprechenden Schaltflächen angezeigt um die Benutzeraccounts zu aktivieren.
			\end{itemize}

		\subsection{Verwaltung}
			\begin{itemize}
				\item \Func{Statistiken anzeigen} \hypertarget{statistik}
					Der Systemadministrator hat die Möglichkeit sich Statistiken bezüglich der Einnahmen anzusehen. Er kann dafür aus 	verschiedenen Parametern wählen, welche Statistik angezeigt werden soll. Die Auswahl der Parameter erfolgt über eine Drop-Down - Liste. Vorgesehene Parameter für die Statistik sind Einnahmen pro 'Tag', pro 'Woche', pro Kursleiter', pro 'Kurs'. Bei Wahl eines der Parameter 'Tag', 'Woche' oder 'Kursleiter' ist ein graphische Darstellung in Form eines Säulendiagramms vorgesehen. Bei Parameter 'Tag' werden die Einnahmen der letzten sieben Tage angezeigt, bei Parameter 'Woche' die Einnahmen der letzten vier Wochen. Bei Wahl des Parameters 'Kurs' ist aufgrund der Menge von Kursen eine tabellarische Anzeige vorgesehen. Durch Betätigen der Schaltfläche 'Anzeigen' wird die gewählte Statistik angezeigt.
				\item \Func{Guthabenkonto aufladen} \hypertarget{guthabenAuf}
					Der Systemadministrator hat die Möglichkeit das Guthabenkonto von  registrierten Nutzern manuell aufzuladen. Damit kann er \glqq Offline - Aufladungen\grqq  des Guthabenkontos eines registrierten Nutzers auf dessen Konto im System weitergeben. Für die Kontoaufladung gibt der Administrator auf der Administratorseite die Identifikationsnummer des Benutzers, dessen Namen und den aufzuladenden Betrag ein. Durch Betätigen der Schaltfläche 'Konto aufladen' wird das Konto des Benutzers mit der eingegebenen Identifikationsnummer um den eingegebenen Betrag aufgeladen. Ab diesem Zeitpunkt kann der registrierte Benutzer das Guthaben verwenden. Das Aufladen des Guthabenkontos ist nur dann möglich, wenn der eingegebene Name auch zur Identifikationsnummer passt. So wird sichergestellt, dass durch Tippfehler bei der Identifikationsnummer nicht der falsche Nutzer das Guthaben erhält.
				\item \Func{Überziehungskredit festlegen.}
					Der Systemadministrator kann festlegen, ob und gegebenenfalls um welchen Betrag ein registrierter Benutzer sein Guthabenkonto überziehen darf. Der Überziehungskredit kann nur global für alle Nutzer festgelegt werden. Dafür trägt der Systemadministrator auf der Administratorseite den Betrag, der maximal überzogen werden darf in das dafür vorgesehene Eingabefeld ein und bestätigt die Eingabe durch Betätigen der 'Speichern'-Schaltfläche.
					Falls kein Überziehungskredit gewährt werden soll, wird als Betrag einfach 0,00 eingegeben. Dieser Wert ist auch per Default gesetzt.
			\end{itemize}
			
		\subsection{Systemanpassung}
			\begin{itemize}
				\item \Func{Logo hochladen.} \hypertarget{logo}
					Das Logo kann vom Systemadministrator editiert werden. Durch Klicken der 'Durchsuchen' - Schaltfläche auf der Administratorseite gibt der Administrator den Pfad einer .jpg - Datei an, welche durch Betätigen der 'Speichern' - Schaltfläche hochgeladen wird. Anforderungen an die .jpg - Datei bezüglich Größe und Auflösung werden während der Produktentwicklung festgelegt.
				\item \Func{Art der Accountaktivierung festlegen.} \hypertarget{regMod}
					Die Art der Aktivierung eines Accounts eines neuen Benutzers kann vom Systemadministrator festgelegt werden. Minimale Voraussetzung für die Aktivierung eines neuen Benutzeraccounts ist die Verifizierung der E-Mailadresse durch den Benutzer. Dafür betätigt der neue Benutzer den Verifizierungslink in der an die eingegebene E-Mailadresse gesandte Bestätigungsmail(vgl. \FuncBlue{/F10-20/}). Des Weiteren kann der Administrator festlegen, dass neue Benutzeraccounts zusätzlich manuell durch einen Kursleiter und/oder einen Administrator aktiviert werden müssen. Dafür wählt der Administrator auf der Administratorseite die entsprechende Option aus der Drop-Down-Liste und betätigt die 'Speichern' - Schaltfläche. Vorgesehene Auswahloptionen sind 'Nur E-Mailverifikation', 'E-Mailverifikation und Aktivierung durch Administrator', 'E-Mailverifikation und Aktivierung durch Kursleiter', 'E-Mailverifikation und Aktivierung durch Administrator oder Kursleiter'.
				\item \Func{Oberfläche editieren.}
					Die Oberfläche kann vom Systemadministrator editiert werden. Durch Klicken der 'Durchsuchen' - Schaltfläche gibt der Administrator den Pfad einer \gls{CSS} - Datei an, welche durch Betätigen der 'Speichern' - Schaltfläche hochgeladen wird. Die ausgewählte Datei muss den Dateinamen 'customStyle.css' haben. In dieser Datei kann zu bestimmten Tags ein neuer Stil definiert werden. 
					Die genauen Tags werden während der Produktentwicklung festgelegt. Der Administrator kann also die Oberfläche bezüglich bestimmter Punkte anpassen.  
					Falls keine 'customStyle.css' - Datei hochgeladen wird oder einige Tags in dieser Datei nicht definiert werden, wird ein im System definierter Default - Stil verwendet.
				\item \Func{Impressum editieren.} \hypertarget{impressumBea}
					Das Impressum kann vom Systemadministrator editiert werden. Durch Klicken der 'Impressum bearbeiten' - Schaltfläche wird der Administrator auf die 'Impressum bearbeiten' - Oberfläche weitergeleitet. Dort wird das Impressum in einem Eingabefeld angezeigt. Der Administrator kann nun das Impressum editieren. Durch Betätigen der 'Speichern' - Schaltfläche werden die Änderungen gespeichert.
			\end{itemize}
		
\resetAllCounter
\renewcommand{\Func}[1]{\stepcounter{Lc}\textcolor{Blue}{\textbf{/D\arabic{Hc}0-\arabic{Lc}0/} #1} \\}	

\chapter{Produktdaten}
 \label{Produktdaten}
        
	
    \section{System}
	     \Func {} 
	     \begin{itemize}
	     	\item Titel (\gls{String})
	     	\item CSS-Datei (String)
	     	\item Impressum (String)
	     	\item Logopfad (String)
	     	\item AGB (String)
	     	\item Hilfe (String)
	     \end{itemize}
	    
	    
	    \subsection{SMTP-Server}
		    \Func {}
		    \begin{itemize}
		    	\item Benutzername (String)
		    	\item Passwort (String)
		    	\item Authentifizierung (\gls{Boolean})
		    	\item TLS (Boolean)
		    	\item Host (String)
		    	\item Port (\gls{Integer})
		    \end{itemize}
	    
	    
\stepHc		    	
	\section{Kurs}
	   \Func {} 
		 \begin{itemize}
		   	\item ID (Integer)
		   	\item Titel (String)
		   	\item Beschreibung (String)
		   	\item Trainer (BenutzerID)
		   	\item nächster Termin (Date)
		   	\item Startdatum (Date)
		   	\item Enddatum (Date)
		   	\item Ort (String)
		   	\item Maximale Teilnehmerzahl (Integer)
		   	\item Bildverweis (String)
		   	\item Kurseinheiten (KursIDs)
		   	\item Teilnehmer (BenutzerIDs)
	    \end{itemize}
\stepHc		    
	\section{Kurseinheit}
	 \Func {} 
		 \begin{itemize}
		 	\item ID (Integer)
		 	\item Titel (String)
		 	\item Beschreibung (String)
		 	\item Trainer (BenutzerID)
		 	\item Beginn (Date)
		 	\item Ende (Date)
		 	\item Ort (String)
		 	\item Preis (\gls{float})
		 	\item Maximale Teilnehmerzahl (Integer)
		 	\item Minimale Teilnehmerzahl (Integer)
		 	\item Teilnehmer (BenutzerIDs)
		 \end{itemize}   
    
    
\stepHc	   		
    \section{Registrierte Benutzer}
		  \Func {} 
		 \begin{itemize}
			\item ID (Integer)
		  	\item Benutzername (String)
		  	\item Passwort (String)
		  	\item Vorname (String)
		  	\item Nachname (String)
		  	\item Geburtsdatum (Date)
		  	\item E-mailadresse (String)
		  	\item Addresse (String)
		  	\item Guthaben (float)
		  	\item Bildverweis (String)
		  	\item eingeschriebene Kurse (KursIDs)
		  	\item eingeschriebene Kurseinheiten (KurseinheitenIDs)
		  	\item eigene Termine (TermineIDs)
		  \end{itemize}

\stepHc		    
	  \section{eigene Termine}
	    \Func {} 
	    \begin{itemize}
	    	\item ID (Integer)
	    	\item Name (String)
	    	\item Beginn (Date)
	    	\item Ende (Date)
	    	\item Beschreibung (String)
	    \end{itemize}
	   
\stepHc	 
	  \section{Benutzerrolle}
	    \Func {} 
	    \begin{itemize}
	    	\item ID (Integer)
	    	\item Gruppenname (String)
	    	\item Benutzer-ID (BenutzerIDs)
	    \end{itemize}
	    
	   
\resetAllCounter
\renewcommand{\Func}[1]{\stepcounter{Lc}\textcolor{Blue}{\textbf{/L\arabic{Hc}0-\arabic{Lc}0/} #1} \\}	    
    
\chapter{Produktleistungen}
	

	\section{Produktleistungen auf Serverseite}
		\begin{itemize}
			\item \Func{} Zur Speicherung der \gls{persistent} Daten wird eine PostgreSQL Datenbank verwendet
			\item \Func{} E-Mailbenachrichtigung, falls von Benutzer erwünscht, bei wichtigen Ereignissen (z.B Registrierung, Buchung einer Kurseinheit bzw Kurses, Terminänderung)
			\item \Func{} Das System hat Wartungszeit von 1,6 Stunden in der Woche bzw. 7,2 Stunden im Monat.
		\end{itemize}
		
\stepHc		
	\section{Produktleistungen auf Clientseite}
			\begin{itemize}
				\item \Func{} Bei Fehleingabe werden bereits eingetragene Daten wieder angezeigt
				\item \Func{} Fehlermeldungen werden im Browser ausgegeben
				\item \Func{} Passwörter werden \gls{hashen}.
				\item \Func{} Übertragung und Einsicht aller persönlichen Informationen erfolgen über eine SSL Verbindung
				\item \Func{} Die Administratoren sind in der Lage registrierte Benutzer zu bearbeiten und ihnen neue Funktionen zuweisen (Trainer oder Administratoren)
				\item \Func{} Die Administratoren haben über eine selbst geschriebene CSS Datei Einfluss auf die Optische Darstellung der Software 
				\item \Func{} Unautorisierte Benutzeraktion (z.B anonymer User will sich für Kurs eintragen) wird nicht angezeigt, stattdessen wird auf die Registrierung bzw. Login verwiesen
				\item \Func{} An einem Client-Browser können nicht mehrere autorisierte verschiedene Sessions zeitgleich geführt werden
				\item \Func{} Verfügbarkeit der Räumlichkeit wird nicht unterstützt.
				\item \Func{} Die Sprache des Systems kann auf Englisch oder Deutsch gesetzt werden. Benutzereingaben werden nicht berücksichtigt und werden unverändert angezeigt.
			\end{itemize}
		
		
 
\chapter{Benutzeroberfläche}
	\begin{tiny}
		PC
	\end{tiny}
   
    \section{Screenshots}
    
	    \subsection{}
    
	    \subsection{}
    
    \section{Navigation}
	    In den Rechtecken der folgenden Grafik sind die verschiedenen Content-Bereiche der Webanwendung enthalten.
		    \newline
		    \newline
%			    \includegraphics[scale=0.25]{logo/Navigation.pdf}
		    \newline
    
	    Die Bereiche sind  nach Farben aufgeteilt (entsprechend der Autorisierung des Nutzers), sodass der jeweilige Bereich
	    \begin{itemize}
		    \item Rot: für alle Nutzer,
		    \item Grün: nur für anonyme Nutzer (bei 'Registrierung' auch für Administratoren),
		    \item Blau: nur für angemeldete Nutzer,
		    \item Violett: nur für Kursleiter und Administratoren,
		    \item Gelb: nur für Administratoren
	    \end{itemize}
	    erreichbar ist.
	    \newline
	    Nach dem erfolgreichen Login wird der Nutzer automatisch auf 'Meine Kurse' weitergeleitet.
	    \newline
	    Durch Ausloggen wird der Nutzer automatisch zur Startseite navigiert.
	    \newline
	    Die Content-Bereiche 'Impressum', 'Hilfe' sowie die Startseite sind global erreichbar für alle Nutzer.
	    \newline
	    Für angemeldete Nutzer sind 'Suche', 'Meine Kurse', 'Profil' sowie 'Terminplaner' über eine Navigationsleiste von jedem Dialog aus erreichbar.
	    \newline
	    Bei Administratoren befindet sich in der Navigationsleiste zusätzlich ein Link zum Bereich 'Adminverwaltung'.
        
    \section{Bedienung}       
	    \begin{itemize}
			\item /B10/ Standardmäßig ist eine menüorientierte Bedienung vorgesehen
			\item /B20/ Die Bedienungsoberfläche ist auf Maus- und Tastaturbedienung ausgelegt
			\item /B30/ Fensterlayout, Dialogstruktur und Mausbedienung entsprechen dem Windows-Gestaltungs-Regelwerk
			\item /B40/ Sämtliche datenschutzrechtlich relevante Daten sind passwortgeschützt
			\item /B50/ Die Darstellung des Fensterinhalts passt sich der Fenstergröße an
		\end{itemize}
            
            
         
\chapter{Qualitätsbestimmungen}
	\begin{tiny}
		RS
	\end{tiny}
	Im Folgenden wird aufgeführt auf welche Qualitätsmerkmale der Webanwendung Wert gelegt wird. Diese orientieren sich an der Norm \textit{ISO/IEC~9126}, welche ein Modell zur Sicherstellung von Softwarequalität darstellt. In wie weit die verschiedenen Qualitätsmerkmale Einfluss auf die Webanwendung nehmen sollen, wird in der unten gezeigten Tabelle definiert.\\

	\begin{table}[h]
 
	    \begin{center}
		    \begin{tabular}{|l|c|}
			    \hline 
			    \rule[-1ex]{0pt}{2.5ex} \textbf{Qualitätskriterium} & \textbf{Stellenwert} \\ 
			    \hline 
			    \rule[-1ex]{0pt}{2.5ex} Funktionalität & sehr wichtig \\ 
			    \hline 
			    \rule[-1ex]{0pt}{2.5ex} Zuverlässigkeit & sehr wichtig \\ 
			    \hline 
			    \rule[-1ex]{0pt}{2.5ex} Benutzbarkeit & sehr wichtig \\ 
			    \hline 
			    \rule[-1ex]{0pt}{2.5ex} Effizienz & wichtig \\ 
			    \hline 
			    \rule[-1ex]{0pt}{2.5ex} Änderbarkeit & weniger wichtig \\ 
			    \hline 
			    \rule[-1ex]{0pt}{2.5ex} Übertragbarkeit & weniger wichtig \\ 
			    \hline   
		    \end{tabular}  
	    \end{center}
	    \caption{Qualitätsmerkmale und ihre zugewiesene Priorität}
	\end{table}

Der Stellenwert der Qualitätskriterien begründet sich wie folgt:
    	\begin{itemize}
    		\item Funktionalität:
    		\item Zuverlässigkeit:
    		\item Benutzbarkeit:
    		\item Effizienz:
    		\item Änderbarkeit:
    		\item Übertragbarkeit:
    	\end{itemize}
	Darüber hinaus sollten noch folgende Qualitätsmerkmale genannt werden, welchen ein besonders hoher oder ein, zur Abgrenzung, geringer Stellenwert zugesprochen wird:
		\begin{itemize}
		 	\item Die Verwendung des Systems soll für den Benutzer einfach und intuitiv möglich sein, weshalb die Benutzbarkeit als sehr wichtig einzustufen ist.
		 	\item Die Benutzung soll komplett im Browser möglich sein, wobei jedoch die Kompatibilität mit älteren Browsern als definiert nicht berücksichtigt wird.
		 	\item Personenbezogene Daten und im Speziellen die Bankverbindung sind vor unberechtigtem Zugriff von außen zu schützen.
		 	\item Geldtransaktionen sind ebenfalls besonders zu schützen, vor allem die \gls{Konsistenz} aller Bezahlaktionen.
 \end{itemize}
 
\resetAllCounter
\newcommand{\Test}[1]{\stepcounter{Lc}\textcolor{Brown}{\textbf{/T\arabic{Hc}0-\arabic{Lc}0/} #1} \\}
\newcommand{\RefFuncBlue}[1]{\textcolor{Blue}{\textbf{#1}}}
\newcommand{\RefFuncGreen}[1]{\textcolor{Green}{\textbf{#1}}}
\newcommand{\RefFuncBrown}[1]{\textcolor{Brown}{\textbf{#1}}}
\chapter{Globale Testszenarien und Testfälle}
 

	\section{Testfälle für den Systemadministrator ohne bestehenden Datensatz}
		\subsection{Setup}
			\begin{itemize}
				\item \Test{Installation} 
			     Der Systemadministrator startet das Installationsprogramm. 
						
				\item Nach dem Start loggt sich der Administrator auf der Startseite über die Schaltfläche 'Anmelden' in dem sich öffnenden Drop-Down Menü mit dem vorgefertigten Benutzername 'admin1' und dem Passwort 'ersterAdmin1' ein und wird nach dem Drücken des sich im Drop-Down Menü befindenden Buttons 'Anmelden' auf die Seite 'Meine Kurse' weitergeleitet. 
				 
				\item \Test{Impressum bearbeiten}
				 Der Systemadministrator befindet sich auf der Seite 'Adminverwaltung', welche er über die Navigationsleiste ausgewählt hat, und klickt dort auf die Schaltfläche 'Impressum bearbeiten' und wird auf diese weitergeleitet. Auf der Seite 'Impressum' fügt er seinen Impressumstext in das Eingabefeld ein, in welchem als Default bereits ein vorgefertigtes Impressum der Firma ofCourse angegeben ist. Anschließend bestätigt der Admin die Eingabe durch Drücken des Button 'Speichern'. Mit dem Bestätigen der Änderungen am Impressum wird der Administrator wieder auch die Seite 'Adminverwaltung' geleitet. (vgl. \RefFuncBlue{/F40-180/})
				 
				
			\end{itemize}		
				
		\subsection{Erstellung und Verwaltung}
			\begin{itemize}
				\item \Test{Überziehungskredit festlegen} 
				 Auf der Seite 'Adminverwaltung' legt der Systemadministrator den Überziehungskredit auf 10 Euro fest, indem er im Textfeld den bereits genannten Betrag '10,00' eingibt und diesen durch Drücken der Schaltfläche 'Speichern' bestätigt. Vor dieser Änderung wurde im Textfeld der Default-Wert von '0,00' angezeigt, nun zeigt dieses Feld den Wert '10,00' an. Der Administrator befindet sich weiterhin auf der Seite 'Adminverwaltung'. (vgl. \RefFuncBlue{/F40-140/})
				
				\item \Test{Art der Accountaktivierung festlegen}
				 Der Administrator befindet sich auf der Seite 'Adminverwaltung'. Im Bereich 'Accountaktivierung' wird im Eingabefeld die voreingestellte minimale Voraussetzung 'nur E-Mailverifikation' angezeigt. Dort wählt der Admin nach dem Klicken auf dieses Eingabefeld im sich öffnenden Drop-Down Menü die Aktivierungsart 'E-Mailverifikation und Aktivierung durch Administrator' aus und bestätigt seine Auswahl durch Drücken des Button 'Speichern'. Im Eingabefeld wird nun die ausgewählte Art 'E-Mailverifikation und Aktivierung durch Administrator' angezeigt. Der Administrator befindet sich weiterhin auf der Seite 'Adminverwaltung'. (vgl. \RefFuncBlue{/F40-160/})
				
				\item \Test{neuen Benutzer anlegen} 
				 Auf der Seite 'Adminverwaltung' klickt der Administrator auf die Schaltfläche 'Benutzer anlegen' und wird auf die Seite 'Benutzer anlegen' weitergeleitet. Hier gibt der Admin folgende Daten ein:
					\begin{itemize}
						\item Benutzername: 'Basti3' 
						\item Passwort: "poIuZt7R'
						\item Passwort: "poIuZt7R'
						\item E-Mail: 'sebastian@nrschwarz.de'
						\item Anrede: 'Herr'
						\item Name: 'Schwarz'
						\item Vorname: 'Sebastian'
						\item Geburtstag: '04.12.1990'
						\item Straße: 'Musterstraße 1'
						\item Postleitzahl: '99999'
						\item Ort: 'Musterstadt'
					\end{itemize}
				 Als Rolle wird 'Kursleiter' festgelegt. Die BenutzerID wird automatisch generiert. Über den Button 'Durchsuchen' unter dem Dummy-Bild wählt der Admin ein Bild des Dateiformates .jpg aus seinem Verzeichnis aus. Dieses erscheint nun anstelle des Dummy-Bildes. Durch Klicken des Buttons 'Benutzer anlegen' bestätigt der Admin die Daten und der neue Benutzer ist angelegt. Der Administrator wird anschließend auf die Seite 'Adminverwaltung' weitergeleitet. (vgl. \RefFuncBlue{/F40-50/})
				
				\item \Test{neuen Benutzer anlegen} 
				 Auf der Seite 'Adminverwaltung' klickt der Administrator auf die Schaltfläche 'Benutzer anlegen' und wird auf die Seite 'Benutzer anlegen' weitergeleitet. 
					 \begin{itemize}
					 \item Hier gibt der Admin folgende Daten ein:
						\begin{itemize}
							\item Benutzername: 'Ricky1' 
							\item Passwort: 'vBn§'
							\item Passwort: 'vBn§'
							\item E-Mail: 'ricky.strohmeier@gmx.de'
							\item Anrede: 'Herr'
							\item Name: 'Strohmeier'
							\item Vorname: 'Ricky'
							\item Geburtstag: '40.10.1991'
							\item Straße: 'Musterstraße 2'
							\item Postleitzahl: '99999'
							\item Ort: 'Musterstadt'
						\end{itemize}
					 Als Rolle wird 'Benutzer' festgelegt und im Bereich Profilbild bleibt das Dummy-Bild bestehen. Da der 40.10.1991 nicht existiert wird nach dem Drücken der Schaltfläche 'Benutzer anlegen' eine Fehlermeldung ausgegeben, welche den Benutzer auf das falsche Datum hinweist und das entsprechende Eingabefeld geleert. Aufgrund des zu kurzen Passwortes und der Verwendung des unerlaubten Zeichens '§' wird der Benutzer des Weiteren mit einer Fehlermeldung darauf hingewiesen, dass sein Passwort nicht akzeptiert wird. Die beiden Passworteingabefelder werden ebenfalls geleert.
					 
					 \item Der Benutzer trägt nun im Eingabefeld des Geburtsdatums den 04.10.1991 ein. Außerdem gibt er ein anderes Passwort im ersten Passwortfeld ein: 'vBnMk89po'. Im zweiten Passwortfeld trägt er 'vBnMk89pO' ein. Nach dem Drücken der Schaltfläche 'Benutzer anlegen' wird der Benutzer mit einer Fehlermeldung darauf hingewiesen, dass seine Passwörter nicht übereinstimmen. Die beiden Passwortfelder werden wieder geleert.
					 
					 \item Der Benutzer gibt nun in beiden Passwortfeldern das Passwort 'vBnMk89po' ein. Durch Klicken des Buttons 'Benutzer anlegen' bestätigt der Admin die Daten und der neue Benutzer ist angelegt. Der Administrator wird anschließend auf die Seite 'Adminverwaltung' weitergeleitet.
					\end{itemize}
				(vgl. \RefFuncBlue{/F40-50/})
					
				\item \Test{neuen Kurs anlegen}
			 	 Der Administrator befindet sich auf der Seite 'Adminverwaltung' und klickt auf die Schaltfläche 'Kurs anlegen'. Dadurch wird er auf die Seite 'Neuen Kurs anlegen' weitergeleitet.
				 	 \begin{itemize} 
					 	 \item Dort trägt er folgende Daten ein:
							\begin{itemize}
								\item Kursname: 'Yoga'
								\item Beschreibung: 'Bringe Körper, Geist und Seele in Einklang durch Yogaübungen. Jeder Teilnehmer sollte seine eigene Turnmatte mitnehmen.'
								\item Startzeitpunkt: '27.04.2015'
								\item Endzeitpunkt: '27.04.2016'
								\item minimale Teilnehmerzahl: '5'
							\end{itemize}
							Nach Drücken der Schaltfläche 'Kurs anlegen' wird der Administrator durch eine Fehlermeldung darauf hingewiesen, dass kein einziger Kursleiter eingetragen wurde.
							
						 \item Der Administrator trägt nun im Bereich 'Kursleiter hinzufügen' den Benutzernamen, die E-Mailadresse und die KursleiterID des Kursleiters 'Basti3' ein. (Daten vgl. \RefFuncBrown{/T10-50/}) Durch Klicken des Buttons 'Kurs anlegen' legt der Admin den neuen Kurs 'Yoga' an und wird anschließend auf die Seite 'Adminverwaltung' weitergeleitet. Die KursId wird automatisch vom System generiert. (vgl. \RefFuncBlue{/F40-10/})
					\end{itemize}
					
				\item \Test{neuen Kurs anlegen}
				 Der Administrator befindet sich auf der Seite 'Adminverwaltung' und klickt auf die Schaltfläche 'Kurs anlegen'. Dadurch wird er auf die Seite 'Neuen Kurs anlegen' weitergeleitet.
					 \begin{itemize}
						\item Dort trägt er folgende Daten ein:
							\begin{itemize}
								\item Name: 'Standardtanz'
								\item Startzeitpunkt: '04.05.2015'
								\item Endzeitpunkt: '33.06.2015'
								\item minimale Teilnehmerzahl: '2'
							\end{itemize}
						Im Bereich 'Kursleiter hinzufügen' trägt der Administrator den Benutzernamen, die E-Mailadresse und die KursleiterID des Kursleiters 'Basti3' ein. (Daten vgl. \RefFuncBrown{/T10-50/}) Da der 33.06.2015 nicht existiert, wird der Administrator nach dem Drücken des Button 'Kurs anlegen' durch eine Fehlermeldung auf seine falsche Eingabe hingewiesen. Das Feld für den Endzeitpunkt wird geleert. 
						
						\item Der Administrator ändert den Endzeitpunkt in dem entsprechenden Eingabefeld auf den 30.06.2015 und fügt folgende Beschreibung in dem dafür vorgesehenen Eingabefeld hinzu: 'Grundkurs Standardtanz. Erlernt werden Tänze wie Walzer, Disco Fox, Cha-Cha-Cha, Slow Fox und Tango. Die Teilnahme als Paar ist wünschenswert.' Durch Klicken des Buttons 'Kurs anlegen' legt der Admin den neuen Kurs 'Standardtanz' an. Der Administrator wird anschließend auf die Seite 'Adminverwaltung' weitergeleitet. Die KursId wird automatisch vom System generiert. (vgl. \RefFuncBlue{/F40-10/}) 		
				 
					\end{itemize}
				 			
			\end{itemize}			
								
		\subsection{Systemdarstellung}
			\begin{itemize}
				 \item \Test{Logo hochladen} 
				 Auf der Seite 'Adminverwaltung' ändert der Systemadministrator das Logo, indem er im Bereich 'Logo ändern' auf 'Durchsuchen' klickt.
					\begin{itemize}
						\item Nach Auswahl der Musikdatei 'Test.mp3' drückt der Admin die Schaltfläche 'Speichern'. Durch eine Fehlermeldung wird der Administrator darauf hingwiesen, dass nur JPG-Dateien akzeptiert werden. Das vorherige Logo bleibt unverändert.
						
					 	\item Nach Auswahl der Bilddatei 'logo2.jpg' drückt der Admin die Schaltfläche 'Speichern', wodurch das Bild hochgeladen wird. Der Administrator befindet sich weiterhin auf der Seite 'Adminverwaltung'. Nach erneutem Laden dieser Seite wird das neue Logo auf der Oberfläche angezeigt.
					\end{itemize}
				 (vgl. \RefFuncBlue{/F40-150/})	
				
				 \item \Test{Oberfläche editieren}
				 Der Systemadministrator befindet sich auf der Seite 'Adminverwaltung' und klickt im Bereich 'Oberfläche anpassen' auf den Button 'Durchsuchen'. 
					\begin{itemize}
						 \item In dem sich öffnenden Fenster wählt der Admin die Datei mit dem Namen 'aenderung.css' aus und bestätigt durch Drücken der Schaltfläche 'Speichern'. Da dieser Dateiname nicht genehmigt ist wird bei erneutem Laden der Seite 'Adminverwaltung' der im System vordefinierte Default-Stil als Oberfläche angezeigt.
						 
						 \item Der Admin wählt erneut die Schaltfläche 'Durchsuchen' im Bereich 'Oberfläche anpassen' aus. In dem sich öffnenden Fenster wählt der Administrator nun die Datei mit dem Namen 'customStyle.css' aus. Nach dem Drücken des Buttons 'Speichern' wird diese Datei hochgeladen. Durch Drücken der Taste 'F5' wird die aktuelle Seite 'Adminverwaltung' neu geladen und die Änderungen durch die gewählte .css-Datei wird sichtbar. 
					\end{itemize}
				 (vgl. \RefFuncBlue{/F40-170/})
				 
			\end{itemize}
			
	\section{Testfälle für den Kursleiter}
		\stepHc
		\begin{itemize}
			\item \Test{eigenen Kurs editieren} 
			 Der Kursleiter 'Basti3' befindet sich auf der Seite 'Meine Kurse' und klickt in seinen Trainingskursen den Kurs 'Yoga' an. Dadurch wird er auf die 'Kursdetails'-Seite weitergeleitet, auf dem bereits die durch den Systemadministrator angegebenen Daten eingetragen sind. (vgl. \RefFuncBrown{/T10-70/}) Der Kursleiter drückt auf den Button 'Bearbeiten', um die Kursdaten zu ändern:
				\begin{itemize}
					\item Der Button 'Bearbeiten' ändert sich nun in den Button 'Speichern' und der Kursleiter kann als neues Startdatum in das dafür vorgesehene Eingabefeld den '31.05.2015' eintragen. Nach dem Drücken der Schaltfläche 'Speichern' wird der Kursleiter durch eine Fehlermeldung darauf hingewiesen, dass das Datum falsch ist. Das Eingabefeld 'Startdatum' wird geleert.
					
				 	\item Der Kursleiter ändert den Startzeitpunkt des Kurses nun auf den '04.05.2015' und bestätigt seine Änderung durch Drücken des Button 'Speichern'. Das neue Datum wird nun im entsprechenden Feld angezeigt und der Button 'Speichern' ändert sich wieder zum Button 'Bearbeiten'. Der Kursleiter bleibt weiterhin auf der Seite 'Kursdetails'. (vgl. \RefFuncBlue{/F30-20/})
				\end{itemize}
				
			\item \Test{Kurseinheiten anlegen} 
			 Der Kursleiter 'Basti3' befindet sich auf der Seite 'Meine Kurse' und klickt in seinen Trainingskursen den Kurs 'Standardtanz' an. Dadurch wird er auf die 'Kursdetails'-Seite weitergeleitet auf dem bereits die durch den Systemadministrator angegebenen Daten eingetragen sind. (vgl. \RefFuncBrown{/T10-80/}) Nach Drücken des Button 'Kurseinheit anlegen' wird er auf die Seite 'Kurseinheit bearbeiten' weitergeleitet. Nach Drücken des Button 'Bearbeiten', welcher sich dadurch zum Button 'Speichern' ändert, können die einzelnen Eingabefelder bearbeitet werden und der Kursleiter legt folgende Kurseinheit an:
				\begin{itemize}
					\item Termin: '04.05.2015'
					\item Ort: 'Turnhalle'
					\item Beschreibung: 'Grundkurs Standardtanz. Erlernt werden Tänze wie Walzer, Disco Fox, Cha-Cha-Cha, Slow Fox und Tango. Die Teilnahme als Paar ist wünschenswert.'
					\item Preis: '10,00'
					\item Kursleiter: 'Basti3'
					\item Mindestteilnehmerzahl: '2'
					\item Maximale Teilnehmerzahl: '10'
					\item KursId: 'autogenerierte Id des Kurses'
				\end{itemize}	
			 Anschließend bestätigt der Kursleiter die Kurseinheit durch Drücken des Buttons 'Speichern', welcher sich dadurch wieder zum Button 'Bearbeiten ändert. Der Kursleiter wird auf die Seite 'Kursdetails' weitergeleitet, auf der im Bereich 'Kurseinheiten' nun die eben angelegte Kurseinheit angezeigt wird. (vgl. \RefFuncBlue{/F30-40/})
			 
			 TO DO: NOCH MEHR EINHEITEN ANLEGEN!!!!
			 
			
			\item \Test{Kurseinheiten editieren}
			 Der Kursleiter 'Basti3' klickt in der Detailansicht des Kurses 'Standardtanz' den Button 'Bearbeiten' an und wird auf die Seite 'Kurseinheit bearbeiten' weitergeleitet. 'Basti3' ändert die letzte Kurseinheit vom '29.06.2015' auf den '30.06.2015' und bestätigt durch Klicken auf den Button 'Speichern'. (vgl. \RefFuncBlue{/F30-50/})
			
			\item \Test{Kurseinheit löschen}
			 Der Kursleiter wählt auf der 'Kurseinheit bearbeiten'-Seite die Checkbox '30.06.2015' aus. Durch Drücken des Buttons 'Löschen' entfernt der Kursleiter diese Kurseinheit. (vgl. \RefFuncBlue{/F30-60/})			
		\end{itemize}
		
			
	\section{Testfälle für anonymer Benutzer}
			\stepHc
			\begin{itemize}
				\item \Test{Kursangebot ansehen} 
				 Der anonyme Benutzer klickt auf der Startseite auf den Button 'Kursangebot'	und gelangt auf die Seite mit den Kursangeboten. Angezeigt wird zunächst das aktuelle Tagesangebot. In dem Suchfeld gibt er den Kursnamen 'Yoga' ein. Es wird nun nur noch der Yogakurs angezeigt. Nach Doppelklicken auf den Kursnamen kann sich der Benutzer nun die Kursdetails ansehen. (vgl. \RefFuncBlue{/F10-30/})
				
				\item \Test{Registrierung} 
				 Über die Schaltfläche 'Login' gelangt der anonyme Benutzer auf die Loginseite und von hier aus über den Button 'registrieren' auf die nächste Seite mit dem Registrierungsformular. 		
					\begin{itemize}
						\item Der Benutzer gibt bei der Registrierung den Benutzernamen 'Ricky1' ein. Da dieser bereits existiert erhält er nach Drücken des 'registrieren'-Buttons eine Fehlermeldung.	
						\item Der Benutzer gibt bei der Registrierung ein zu kurzes Passwort ein und erhält nach Drücken des 	'registrieren'-Buttons eine Fehlermeldung.	
						\item Der Benutzer gibt zwei verschiedene Passwörter ein und erhält nach Drücken des 'registrieren'-Buttons eine Fehlermeldung.
						\item Der Benutzer hat seine Mailadresse nicht angegeben und erhält nach Drücken des 'registrieren'-Buttons eine Fehlermeldung.	
						\item Der Benutzer registriert sich mit folgenden Daten: 
							\begin{itemize}
								\item Benutzername: 'Kathi5'
								\item Passwort: 'bSdFg7HjK'
								\item Passwort: 'bSdFg7HjK'
								\item E-Mailadresse: 'katharina\_hoelzl@web.de'
								\item Anrede: 'Frau'
								\item Name: 'Hölzl'
								\item Vorname: 'Katharina'
								\item Geburtsdatum: '29.05.1993'
								\item Adresse: 'Am Kastenfeld 39'
								\item Postleitzahl: '94081'
								\item Ort: 'Fürstenzell'
							\end{itemize}
						Der Benutzer wählt außerdem die Option 'E-Mail-Verifizierung' aus der Liste der Verifizierungsarten aus. Nach Drücken des 'registrieren'-Buttons wird eine E-Mail mit dem Aktivierungslink an den Administrator geschickt.
					\end{itemize}
				(vgl. \RefFuncBlue{/F10-20/})
			\end{itemize}	
		
	\section{Testfälle für den registrierten Benutzer}
		\stepHc
		\begin{itemize}
			\item \Test{Account bestätigen durch den Systemadministrator}
			Der Systemadministrator erhält einen Aktivierungslink per Mail um den Benutzer 'Kathi5' zu verifizieren. Er klickt diesen Link an. Dadurch wird der Account von 'Kathi5' aktiviert. (vgl. \RefFuncBlue{/F30-120/})
			
			\item \Test{Einloggen ins System} 
			Der Benutzer 'Kathi5' klickt auf der Startseite	auf den 'Login'-Button und es öffnet sich ein Drop-Down Menü, in welchem er Benutzername und Passwort eingeben kann. (vgl. \RefFuncBlue{/F20-10/})
			\begin{itemize}
				\item Der Benutzer gibt seinen Benutzernamen 'Kathi5' und das falsches Passwort 'wertz6uio' ein. Nach Drücken des 'Login'-Buttons erhält er die Fehlermeldung 'Ihr Benutzername oder Ihr Passwort ist falsch'.
				\item Der Benutzer gibt den falschen Benutzernamen 'Kathi1' und das richtige Passwort 'bSdFg7HjK' ein. Nach Drücken des 'Login'-Buttons erhält er die Fehlermeldung 'Ihr Benutzername oder Ihr Passwort ist falsch'.
				\item Der Benutzer gibt seinen Benutzernamen 'Kathi5' und kein Passwort ein. Nach Drücken des 'Login'-Buttons erhält er die Fehlermeldung 'Ihr Benutzername oder Ihr Passwort ist falsch'.
				\item Der Benutzer gibt seinen Benutzernamen 'Kathi5' und das richtige Passwort 'bSdFg7HjK' ein. Nach Drücken des 'Login'-Buttons wird der Benutzer auf die Seite mit seinen Kursen weitergeleitet.
			\end{itemize} 
			
			\item \Test{Konto aufladen} 
			Der Benutzer 'Kathi5' klickt in seinem Profil auf die Schaltfläche 'Konto aufladen' und wird auf die Seite 'Kontoaufladung' weitergeleitet. Hier gibt Kathi5 folgende Daten ein:
				 \begin{itemize}
				 	\item Benutzerid: 'autogen'
				 	\item Name: 'Hölzl'
				 	\item Vorname: 'Katharina'
				 	\item Kreditinstitut: ???
				 	\item Kreditkartennummer: ???
				 	\item Betrag: 50 
				 \end{itemize}	
			Anschließend klickt der Nutzer auf den Button 'Aufladen' und erhält eine Erfolgsmeldung.  (vgl. \RefFuncBlue{/F20-240/})
			\end{itemize}
			
			\begin{itemize}
				\item \Test{Kurs anmelden} 
				Der Benutzer 'Kathi5' klickt im Register 'Meine Kurse' auf die Schaltfläche 'Weitere Kurse' und bekommt die einzelnen Kurse auf der Kursangebotsseite angezeigt. Im Kursangebot sucht 'Kathi5' nach dem Kurs 'Yoga'. Durch Doppelklicken auf den gefundenen Kurs erhält der Benutzer die Detailansicht des Kurses 'Yoga', welcher regelmäßig am Dienstag von 16 Uhr bis 17:30 Uhr angeboten wird und kostenlos ist. 'Kathi5' trägt sich durch die Auswahl der Checkbox 'Kursnews' ein und erhält so bei Kursveränderungen eine Informationsmail. Anschließend meldet der Benutzer sich durch Klicken auf den Button 'Anmelden' in der Detailansicht für den Kurs 'Yoga' an. (vgl. \RefFuncBlue{/F20-170/})
				
				\item \Test{Kurseinheiten anmelden} 
				Der Nutzer 'Kathi5'	klickt im Register 'Meine Kurse' auf den Kurs 'Yoga' und erhält dessen Detailansicht. Durch Klicken des Buttons 'Alle Einheiten auswählen' und anschließend der Schaltfläche 'Anmelden' trägt sich der Benutzer nun in alle verfügbaren Kurseinheiten ein. (vgl. \RefFuncBlue{/F20-190/})
				
				\item \Test{Kurs bezahlen}
				Der Benutzer 'Kathi5' meldet sich außerdem auf der Kursangebotsseite für den Kurs 'Standardtanz' an. Durch Doppelklick auf den Kurs erhält der Nutzer die Detailansicht und wählt die folgenden Einheiten aus:
					\begin{itemize}
						\item Montag 04.05.2015 18-20 Uhr
						\item Montag 11.05.2015 18-20 Uhr
						\item Montag 25.05.2015 18-20 Uhr		
					\end{itemize}	
				Durch Drücken der Schaltfläche 'Anmelden' trägt sich 'Kathi5' in die ausgewählten Kurseinheiten ein. Da eine Kurseinheit 15 Euro kostet werden automatisch durch das Anmelden 45 Euro von seinem Guthabenkonto abgebucht.
				
				\item \Test{eigenen Termin in den persönlichen Terminplaner eintragen}
				Über das Register 'Terminplaner' lässt sich 'Kathi5' ihren persönlichen Terminplaner anzeigen. Es existieren im Planer bereits der Yogakurs am Dienstag von 16 Uhr bis 17:30 Uhr und der Kurs 'Standardtanz' am Montag von 18 bis 20 Uhr. Der Benutzer trägt unter dem Terminplaner in der Zeile 'neuer Kurs' folgende Daten ein:
					\begin{itemize}
						\item Name: Fußball
						\item Datum: 14.05.2015
						\item Zeit von 17:00 bis 20:00 Uhr
						\item Ort: Trainingsfeld		
					\end{itemize}
				Durch Klicken auf die Schaltfläche 'Eintragen' wird dieser Termin in den persönlichen Terminplaner eingetragen. (vgl. \RefFuncBlue{/F20-130/})
				
				\item \Test{Kurseinheit abmelden und Rückbuchung des Kurspreises}
				Der Benutzer 'Kathi5' klickt in der Detailansicht des Kurses 'Standardtanz' auf die Kurseinheit 'Montag 25.05.2015 18-20 Uhr' und Drückt den Button 'Abmelden'. Dadurch wurde 'Kathi5' aus dieser Kurseinheit ausgetragen und der Betrag von 15 Euro wird automatisch auf das Guthabenkonto rückgebucht. (vgl. \RefFuncBlue{/F20-200/}) und (vgl. \RefFuncBlue{/F20-270/})
				
				\item \Test{aktuelles Guthaben ansehen}
				Der Nutzer 'Kathi5' klickt auf die Registerkarte 'Mein Profil'. Hier wird der aktuelle Kontostand von 20 Euro angezeigt. (vgl. \RefFuncBlue{/F20-70/})
				
				\item \Test{Hilfeseite anzeigen}
				Der Benutzer 'Kathi5' befindet sich momentan auf der Kursdetailseite des Kurses 'Yoga' und lässt sich durch Drücken des Buttons 'Hilfe' die Hilfeseite anzeigen. Dort sucht er die Information zum Thema 'Profilbild hochladen'. (vgl. \RefFuncBlue{/F10-70/})
					
				\item \Test{Hochladen eines Profilbildes}
				Der Nutzer 'Kathi5' klickt auf die Registerkarte 'Mein Profil'. 
					\begin{itemize}
						\item Nach Drücken der Schaltfläche 'Bild Hochladen' wird das Video 'Test.mp3' ausgewählt. Es erscheint eine Fehlermeldung, da eine Videodatei hochgeladen wurde.
						\item Nach Drücken der Schaltfläche 'Bild Hochladen' wird das Bild 'KathiProfil.jpg' ausgewählt und hochgeladen. Es erscheint nun anstelle des Dummy-Bildes.		
					\end{itemize}
				
				Nach Drücken der Schaltfläche 'Bild Hochladen' wird das Bild 'KathiProfil.jpg' ausgewählt und hochgeladen. Es erscheint nun anstelle des Dummy-Bildes. (vgl. \RefFuncBlue{/F20-60/})
				
				\item \Test{E-Mailadresse ändern}
				'Kathi5' klickt auf die Registerkarte 'Mein Profil'. Nach Drücken der Schaltfläche 'Benutzerdaten ändern' ändert sie die E-Mailadresse auf 'hoelzlka@fim.uni-passau.de'. Anschließend klickt sie auf den Button 'Speichern' und erhält an die angegebene Mailadresse einen Verifizierungslink. Nach Bestätigung durch diesen Link wird der Benutzer auf die Seite 'Mein Profil' weitergeleitet, in dem die E-Mailadresse auf die Aktuelle geändert ist. (vgl. \RefFuncBlue{/F20-50/})
				
				\item \Test{aus dem System abmelden}
				Der Benutzer 'Kathi5' befindet sicht momentan auf der Seite 'Mein Profil'. Durch Drücken der Schaltfläche 'Abmelden' wird 'Kathi5' aus dem System abgemeldet und zur Startseite weitergeleitet. (vgl. \RefFuncBlue{/F20-80/})
					
			\end{itemize}
\section{Testfälle mit Datensatz}
			\stepHc
			\begin{itemize}
				\item \Test{Benachrichtigung durch Kursleiter}
				'??? muss noch abgeklärt werden ???' (vgl. \RefFuncBlue{/F30-110/})
				
				\item \Test{Invited Guest durch Kursleiter zu Kurs hinzufügen}
				Der Kursleiter 'Basti3' fügt auf der Seite 'Einheiten bearbeiten' die Daten des neuen Teilnehmers hinzu. (vgl. \RefFuncGreen{/F30-90W/})
						
				\item \Test{Benutzerrolle aufwerten durch Systemadministrator} 
				Der Systemadministrator klickt auf der Seite 'Admin-Verwaltung' auf die Schaltfläche 'Benutzer verwalten' und wird auf die Seite 'Benutzer bearbeiten' weitergeleitet. Hier wählt er im Status des Benutzers 'Kathi5' die neue Rolle 'Kursleiter' aus. Durch Drücken des Buttons 'Speichern' bestätigt der Administrator diese Änderung. (vgl. \RefFuncBlue{/F40-90/})
				
				\item \Test{Teilnehmer aus Kurs entfernen durch Kursleiter}
				Auf der Oberfläche 'Kursangebot' wählt der Kursleiter 'Basti3' durch Doppelklick den Kurs 'Yoga' aus und wird auf die Seite 'Kurs bearbeiten' weitergeleitet. Dort wählt er den Teilnehmer 'Kathi5' aus und entfernt diesen aus dem Kurs durch Klicken der Schaltfläche 'löschen'. (vgl. \RefFuncBlue{/F30-80/})
				
				\item \Test{Kursleiter zu Kurs hinzufügen durch Systemadministrator}
				Auf der Oberfläche 'Kursangebot' wählt der Systemadministrator durch Doppelklick den Kurs 'Yoga' aus und wird auf die Seite 'Kurs bearbeiten' weitergeleitet, auf der er die Daten des Kursleiters 'Kathi5' eingibt und die Änderung durch Drücken des Buttons 'Hinzufügen' abspeichert. 
				
				\item \Test{Statistiken anzeigen}
				Auf der Seite 'Admin-Verwaltung' klickt der Administrator in der Navigationsleiste auf den Link 'Statistik und wird auf diese weitergeleitet. Nach Auswahl des Parameter 'Tag' wird ein Säulendiagramm mit den Einnahmen der entsprechenden Woche angezeigt. (vgl. \RefFuncBlue{/F40-110/})
				
				\item \Test{Benutzer suchen}
				Über die Seite 'Admin-Verwaltung' gelangt der Administrator auf die Seite 'Kursangebot' und gibt dort den entsprechenden Suchbegriff ein. Nachdem der Admin auf den Button 'Suchen' geklickt hat, wird ihm der gesuchte Benutzer angezeigt. (vgl. \RefFuncBlue{/F40-100/})
				
				\item \Test{Guthabenkonto aufladen durch Systemadministrator}
				Der Systemadministrator trägt auf der Seite 'Admin-Verwaltung' im Bereich 'Konto aufladen' folgende Daten ein:
					\begin{itemize}
						\item BenutzerId: 'autogen'
						\item Benutzername: Ricky1
						\item Betrag: 40 Euro
					\end{itemize}
				Durch Drücken des Buttons 'Aufladen' bestätigt der Systemadministrator die Kontoaufladung. (vgl. \RefFuncBlue{/F20-150/})
				
				\item \Test{Kurs löschen durch Systemadministrator}
				Der Systemadministrator wählt auf der Seite 'Kursangebot' den zu löschenden Kurs 'Yoga' durch Doppelklick aus und gelangt dadurch auf die Seite 'Kurse bearbeiten'. Durch Drücken des Buttons 'Kurs löschen' entfernt der Admin den Kurs 'Yoga'.
				
				
				\item \Test{Benutzer löschen durch Systemadministrator}
				Der Systemadministrator wählt auf der Seite 'Benutzer verwalten' den zu löschenden Benutzer 'Ricky1' durch Doppelklick aus und gelangt dadurch auf die Seite 'Benutzer bearbeiten'. Durch Drücken des Buttons 'Benutzer löschen' entfernt der Admin den Benutzer 'Ricky1'. (vgl. \RefFuncBlue{/F40-80/})
				
				\item \Test{Kurs löschen durch Kursleiter}
				Der Kursleiter 'Basti3' auf der Oberfläche mit seinen Trainingskursen den Kurs 'Standardtanz' an und wird auf die Seite 'Kurs bearbeiten' weitergeleitet. Durch Klicken des Buttons 'Kurs löschen' wird der Kurs 'Standardtanz' entfernt. (vgl. \RefFuncBlue{/F40-40/})
				
			\end{itemize}





\chapter{Entwicklungsumgebung}
    \section{Software}
        \subsection{Betriebssysteme}
            \begin{itemize}
            	\item Windows 8.1
            	\item Windows 7
            	\item Linux Mint 17.1 Rebecca
            \end{itemize}	
            
        \subsection{Dokumentation}
            \begin{itemize}
            	\item Latex (u.a. Texmaker 4.1, TeXstudio 2.6.6)
            \end{itemize}
            
        \subsection{Datenbank}
           \begin{itemize}
           	\item PostgreSQL 9.1
           \end{itemize}
           
        \subsection{Webserver}
            \begin{itemize}
            	\item Tomcat 8
            \end{itemize}
            
        \subsection{Entwicklung}
           \begin{itemize}
           	\item IBM Rational Software Architect
           	\item Eclipse Luna 4.4.2
           	\item Eclipse Plugin: \\
	           	\begin{itemize}
	           		\item jUnit
	           		\item EGit
	           		\item Mylyn
	           		\item Checkstyle  
	           	\end{itemize} 
           \end{itemize}
           
        \subsection{sonstige Software}
            \begin{itemize}
            	\item Inkscape
            	\item DB-Visualizer 9.0.6
            	\item Gimp 2.8.4
            \end{itemize}
           
    \section{Orgware}
        \begin{itemize}
        	\item GitHub
        	\item Dropbox
        \end{itemize}
        
    \section{Hardware}
        \begin{itemize}
        	\item Intel Core i5-3317U CPU @ 1.70GHz x 2
        \end{itemize}
        
        
        
\chapter{Ergänzung}
\begin{tiny}
PC
\end{tiny}

	\begin{itemize}
	\item Der HTML 5 Code wird durch den Validator
	http://validator.w3.org überprüft
	\item Es werden keine Frames verwendet
	\end{itemize}        
        
        
        
        
        
\printnoidxglossaries
\end{document}
