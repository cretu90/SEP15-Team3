\documentclass[a4paper]{scrreprt}
\usepackage[german]{babel}
\usepackage[german]{translator}
\usepackage[utf8]{inputenc}
\usepackage[T1]{fontenc}
\usepackage{blindtext} 
\usepackage{ae}
\usepackage[bookmarks,bookmarksnumbered]{hyperref}
\usepackage{graphicx}
\usepackage{color}
\usepackage[dvipsnames]{xcolor}

%Glossar
\usepackage[nonumberlist, toc, section=chapter, numberedsection=nolabel]{glossaries}
\makeglossaries

%Glossareinträge



\begin{document}
	\thispagestyle{plain}

\begin{titlepage}
    \begin{center}
\begin{figure}[th]
\centering
\includegraphics[width=0.6\linewidth]{logo/name_blau.jpg}
\end{figure}

    	\begin{title}
        	\title{\Huge{\textbf{Kurseinheiten Manager \\ Pflichtenheft\\}}}

		\end{title}
		\hspace{3cm}

        	Software Engineering Praktikum \\
        	Sommersemester 2015\\
        	Universität Passau\\


        	Betreuer: Andreas Stahlbauer\\
        	\hspace{1,5cm}\\
        	Version: 1.0 \\
        	\hspace{1,5cm}\\
        	Datum: 17.04.2015\\[50pt]
        	\textbf{Team 3} \\
            \ \\
    
        
        
        \begin{tabular}{ | l | l | l | l |}
            \hline
            \textbf{Matrikelnummer} & \textbf{Name} & \textbf{Phase} & \textbf{E-Mail}  \\ \hline
            63097 & Katharina Hölzl & Pflichtenheft & hoelzlka@fim.uni-passau.de \\ \hline
            64504 & Ricky Strohmeier& Entwurf & strohric@fim.uni-passau.de  \\ \hline
            64380 & Martin Bachhuber & Feinspezifikation  & bachhube@fim.uni-passau.de \\ \hline
            64080 & Tobias Fuchs & Implementierung  &  fuchstob@fim.uni-passau.de\\ \hline
            61085 & Sebastian Schwarz & Validierung & sebastian@nrschwarz.de \\ \hline  
            58379 & Patrick Cretu  &  Präsentation & cretu@fim.uni-passau.de \\ \hline
        \end{tabular}
    \end{center}
\end{titlepage}
 
 


% Platzierung des Inhaltsverzeichnisses
\tableofcontents
 
\chapter{Zielbestimmung}
	   
    \section{Musskriterien}      
    	\subsubsection{Allgemeiner Funktionsumfang:}
      		
     	\subsubsection{Funktionsumfang für anonymen Benutzer:}
       	
     	\subsubsection{Funktionsumfang für registrierten Benutzer:}
		
		\subsubsection{Funktionsumfang für Kursadministrator:}
		
		\subsubsection{Funktionsumfang für Systemadministrator:}
		
		\subsubsection{Funktionsumfang Benutzergruppen:}
			
    \section{Wunschkriterien}
			
		\section{Abgrenzungskriterien}
     		
        
  
\chapter{Produkteinsatz}
    \section{Anwendungsbereiche}
		   
     
	\section{Zielgruppen}
        
    
	\section{Betriebsbedingungen}
       
	
			
 
\chapter{Produktumgebung}
	\section{Software}
        \begin{itemize}
      		\item Client:
      		
          	\item Server:
            
        \end{itemize}
    \section{Hardware}   
        \begin{itemize}
          	\item Client:
            
          	\item Server:
           
        \end{itemize}
     \section{Orgware}
             Es wird keine Orgware benötigt.
 
\chapter{Produktfunktionen}
	
	

	\section{Anonymer Benutzer}
		\subsection{Grundfunktionen}
			\begin{itemize}
				\item {Aufruf der Startseite:}
               \blindtext
			    \item {Registrierung im System durch Ausfüllen eines Registrierungsformulars:}
                \blindtext
			    
			\end{itemize}
		\subsection{Bezahlfunktionen}
			
			\begin{itemize}
			    \item {text} \blindtext
			    
			\end{itemize}
	
	\section{Registrierter Benutzer}
	
	
		\subsection{Grundfunktionen}
			
		\subsection{Kursfunktionen}
			
			\begin{itemize}
			    \item {text} 
                \blindtext
			    \item {text} 
               \blindtext
			\end{itemize}
		\subsection{Benachrichtigungen}
			
			
	
	\section{Kursadministratoren}
		
        
		\subsection{Konfiguration von Kursen}
			\subsubsection{Grundlegende Konfiguration}
				
			\subsubsection{Kursverwaltung}
				
			\subsubsection{Graphische Anpassung, Anzeigeoptionen}
				
		\subsection{Benutzergruppen und Adressbuch}
			
			
		\subsection{Monitoring}
		
			

		
	
	\section{Systemadministratoren}
		
		
		\subsection{Benutzerkonten}
			
		\subsection{Verwaltung der Kursen}
			
			
		\subsection{Systemanpassung}
			
			
			
	\section{Webservice}
		
		

\chapter{Produktdaten}
 \label{Produktdaten}
        
	
    \section{System}
	    
	    
	    
	    \subsection{SMTP-Server}
	    
	    
	    
	    	
	\section{Kurs}
	   
	    
	    
    
    
    		
    \section{Registrierte Benutzer}
	  
	    
	    
	   
	    
    
\chapter{Produktleistungen}
	

	\section{Produktleistungen auf Serverseite}
		
	\section{Produktleistungen auf Clientseite}
	
		
		
 
\chapter{Benutzungsoberfläche}
    
    
    \subsection{Anonymer Benutzer}
       	
       	
       	
    \subsection{Registrierter Benutzer}
       
       	
       	    
    \subsection{Kursadministrator}
        
        

    \subsection{Systemadministrator}
        
        
            
   
    
    \section{Verwaltungsoberfläche des Kursadministrators}
    
   
       
    
    \section{Kurseigenschaften}
    
    
       
    
    
    \section{Navigationsdiagramme}
        \subsection{Anonymer Benutzer}
            
            
            
        \subsection{Registrierter Benutzer}
             
            
            
        \subsection{Kursadministrator}
            
             
            
        \subsection{Systemadministrator}
           
            
            
            
         
\chapter{Qualitätsbestimmungen}

\begin{table}[h]
 
    \begin{center}
    \begin{tabular}{|l|c|}
    \hline 
    \rule[-1ex]{0pt}{2.5ex} \textbf{Qualitätskriterium} & \textbf{Bedeutung} \\ 
    \hline 
    \rule[-1ex]{0pt}{2.5ex} Funktionalität &  \\ 
    \hline 
    \rule[-1ex]{0pt}{2.5ex} Zuverlässigkeit &  \\ 
    \hline 
    \rule[-1ex]{0pt}{2.5ex} Benutzbarkeit &  \\ 
    \hline 
    \rule[-1ex]{0pt}{2.5ex} Effizienz &  \\ 
    \hline 
    \rule[-1ex]{0pt}{2.5ex} Änderbarkeit &  \\ 
    \hline 
    \rule[-1ex]{0pt}{2.5ex} Übertragbarkeit &  \\ 
    \hline   
    \end{tabular}  
    \end{center}
    \caption{+: weniger wichtig, ++: wichtig, +++: sehr wichtig} 
    \label{qTabelle}   
\end{table}
    
Darüber hinaus sollten noch folgende Qualitätsmerkmale genannt werden, die in der Norm nicht berücksichtigt werden:
 
\chapter{Globale Testszenarien und Testfälle}
 

	\section{Testfälle für den Systemadministrator ohne bestehenden Datensatz}
		\subsection{Setup}
				
		\subsection{Erstellung und Verwaltung}
			
			
		\subsection{Systemdarstellung}
			
			
			
		\section{Testfälle für den Kursadministrator}
		
			
	\section{Testfälle für anonymer Benutzer}
	
		
	\section{Testfälle für den registrierten Benutzer}
		

\section{Testfälle mit Datensatz}






\chapter{Entwicklungsumgebung}
    \section{Software}
        \subsection{Betriebssysteme}
            
        \subsection{Dokumentation}
            
        \subsection{Datenbank}
           
        \subsection{Webserver}
            
        \subsection{Entwicklung}
           
        \subsection{sonstige Software}
           
    \section{Orgware}
       
    \section{Hardware}
        
\printglossaries
\end{document}
