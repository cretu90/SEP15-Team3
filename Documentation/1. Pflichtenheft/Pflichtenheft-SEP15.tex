\documentclass[a4paper]{scrreprt}
\usepackage[german]{babel}
\usepackage[german]{translator}
\usepackage[utf8]{inputenc}
\usepackage[T1]{fontenc}
\usepackage{blindtext} 
\usepackage{ae}
\usepackage[bookmarks,bookmarksnumbered]{hyperref}
\usepackage{graphicx}
\usepackage{color}
\usepackage[dvipsnames]{xcolor}

\newcounter{Lc}
\newcounter{Hc}
\newcommand{\stepHc}{\stepcounter{Hc}\setcounter{Lc}{0}}
\newcommand{\resetAllCounter}{\setcounter{Lc}{0}\setcounter{Hc}{1}}

%Glossar
\usepackage[nonumberlist, toc, section=chapter, numberedsection=nolabel]{glossaries}
\makeglossaries

%Glossareinträge



\begin{document}
	\thispagestyle{plain}

\begin{titlepage}
    \begin{center}
\begin{figure}[th]
\centering
\includegraphics[width=0.6\linewidth]{logo/name_blau.jpg}
\end{figure}

    	\begin{title}
        	\title{\Huge{\textbf{Kurseinheiten Manager \\ Pflichtenheft\\}}}

		\end{title}
		\hspace{3cm}

        	Software Engineering Praktikum \\
        	Sommersemester 2015\\
        	Universität Passau\\


        	Betreuer: Andreas Stahlbauer\\
        	\hspace{1,5cm}\\
        	Version: 1.0 \\
        	\hspace{1,5cm}\\
        	Datum: 17.04.2015\\[50pt]
        	\textbf{Team 3} \\
            \ \\
    
        
        
        \begin{tabular}{ | l | l | l | l |}
            \hline
            \textbf{Matrikelnummer} & \textbf{Name} & \textbf{Phase} & \textbf{E-Mail}  \\ \hline
            63097 & Katharina Hölzl & Pflichtenheft & hoelzlka@fim.uni-passau.de \\ \hline
            64504 & Ricky Strohmeier& Entwurf & strohric@fim.uni-passau.de  \\ \hline
            64380 & Martin Bachhuber & Feinspezifikation  & bachhube@fim.uni-passau.de \\ \hline
            64080 & Tobias Fuchs & Implementierung  &  fuchstob@fim.uni-passau.de\\ \hline
            61085 & Sebastian Schwarz & Validierung & sebastian@nrschwarz.de \\ \hline  
            58379 & Patrick Cretu  &  Präsentation & cretu@fim.uni-passau.de \\ \hline
        \end{tabular}
    \end{center}
\end{titlepage}
 
 


% Platzierung des Inhaltsverzeichnisses
\tableofcontents
 
\chapter{Zielbestimmung}
	   
    \section{Musskriterien}      
    	\subsubsection{Allgemeiner Funktionsumfang:}
      		
     	\subsubsection{Funktionsumfang für anonymen Benutzer:}
       	
     	\subsubsection{Funktionsumfang für registrierten Benutzer:}
		
		\subsubsection{Funktionsumfang für Kursadministrator:}
		
		\subsubsection{Funktionsumfang für Systemadministrator:}
		
		\subsubsection{Funktionsumfang Benutzergruppen:}
			
    \section{Wunschkriterien}
			
		\section{Abgrenzungskriterien}
     		
        
  
\chapter{Produkteinsatz}
    \section{Anwendungsbereiche}
		   
     
	\section{Zielgruppen}
        
    
	\section{Betriebsbedingungen}
       
	
			
 
\chapter{Produktumgebung}
	\section{Software}
        \begin{itemize}
      		\item Client:
      		
          	\item Server:
            
        \end{itemize}
    \section{Hardware}   
        \begin{itemize}
          	\item Client:
            
          	\item Server:
           
        \end{itemize}
     \section{Orgware}
             Es wird keine Orgware benötigt.

\resetAllCounter
\newcommand{\Func}[1]{\stepcounter{Lc}\textcolor{Blue}{\textbf{/F\arabic{Hc}0-\arabic{Lc}0/} #1} \\}
\newcommand{\FuncW}[1]{\stepcounter{Lc}\textcolor{Green}{\textbf{/F\arabic{Hc}0-\arabic{Lc}0W/} #1} \\}
\newcommand{\FuncBlue}[1]{\textcolor{Blue}{\textbf{#1}}}
\newcommand{\FuncGreen}[1]{\textcolor{Green}{\textbf{#1}}}
\chapter{Produktfunktionen}
Die Produktfunktionen sind nach Benutzergruppen geordnet. Wunschkriterien sind mit nachgestelltem 'W' gekennzeichnet.

\section{Anonymer Benutzer}
\subsection{Grundfunktionen}
\begin{itemize}
	\item \Func{Aufruf der Startseite.}
	Die Startseite kann mit jedem unterstützen Browser(vgl. Produktumgebung) via URL aufgerufen werden.
	\item \Func{Registrierung im System durch Ausfüllen eines Formulars.}
	Der Nutzer kann sich durch ausfüllen eines Registrierungsformulars im System registrieren. Die Anwendung generiert für den neu registrierte Nutzer automatisch eine im System eindeutige Identifikationsnummer. Nach Abschicken des Formulars gibt es zwei Möglichkeiten um den Account zu aktivieren. Entweder erhält der Nutzer eine E-Mail an die von ihm angegebene Adresse, welche einen Aktivierungslink enthält oder der Account wird vom Betreiber bzw. einem Trainer freigeschaltet. Welche der beiden Möglichkeiten zum Einsatz kommt, ist vom Administrator in seinen Einstellungen für das System auszuwählen.
	\item \Func{Sicheres Einloggen ins System.}
	Durch Eingabe des Benutzernamen,des Passworts und Klicken der 'Anmelden' - Schaltflächen kann sich der Benutzer ins System einloggen. Die Anmeldedaten werden dabei mittels SSL übertragen.
	\item \Func{Anzeigen des Kursangebots.}
	Durch Klicken der Schaltfläche 'Kursangebot' wird dem Nutzer das Kursangebot angezeigt. Als Default wird dabei das Angebot des aktuellen Tages angezeigt. Weitere Anzeigezeiträume sind möglich (vgl. (\FuncBlue{/F10-50/})). Angezeigt werden alle stattfindenden Einheiten der Kurse in dem gewählten Zeitraum.
	\item \Func{Einschränken des angezeigten Kursangebots.}
	Der Nutzer hat die Möglichkeit das angezeigte Kursangebot auf verschiedene Zeiträume einzuschränken. Als Default - Zeitraum ist der aktuelle Tag vorgesehen. Weitere Anzeigezeiträume sind 'Wochenangebot', 'alle Kurse'. Die Auswahl des Anzeigezeitraums ist über eine Drop-Down-Liste möglich. Bestätigt wird die Auswahl durch Klicken der 'Anzeigen' - Schaltfläche.
	\item \Func{Sortieren des angezeigten Kursangebots.}
	Die angezeigten Kurse können entweder nach 'Kursame' alphabetisch oder nach 'Uhrzeit' chronologisch sortiert werden.
	\item \Func{Suche nach Kursen.}
	Das Kursangebot kann durch Eingabe des Kursnamens nach einem bestimmten Kurs durchsucht werden.
	\item \Func{Anzeige des Impressums.}
	Das Impressum kann von jeder Seite des Systems aus angezeigt werden.
	\item \Func{Anzeige der Hilfeseite.}
	Von jeder Seite des Systems kann die Hilfeseite zur entsprechenden Seite des Systems oder die komplette Hilfeseite aufgerufen werden.
	\item \Func{Anzeige der Allgemeinen Geschäftsbedingungen.}
	Die allgemeinen Geschäftsbedingungen können von jeder Seite des Systems aus angezeigt werden.
	\item \FuncW{Wechsel der Anzeigesprache.}
	Von jeder Seite des Systems aus kann durch eine Klicken einer entsprechenden Schaltfläche die Anzeigesprache gewechselt werden. Als Default ist 'Deutsch' ausgewählt. Als weitere Sprache ist 'Englisch' vorgesehen. 
	\item \FuncW{'Passwort vergessen' - Schaltfläche.}
	Klickt der Benutzer die 'Passwort vergessen' - Schaltfläche, so wird er aufgefordert seine im System hinterlegte E-Mailadresse einzugeben. An diese wird dann ein automatisch generiertes Passwort geschickt.
\end{itemize}

\section{Registrierter Benutzer}
Der registrierte Benutzer verfügt über sämtliche Funktionen wie der anonyme Benutzer. Zusätzlich kann er auf folgende Funktionen zurückgreifen.
\stepHc
\subsection{Grundfunktionen}
\begin{itemize}
	\item \Func{Anzeigen der angemeldeten Kurse.}
	Die Kurse zu denen der registrierte Benutzer angemeldet ist werden gesammelt angezeigt.
	\item \Func{Detailanzeige der angemeldeten Kurse.}
	Durch Klicken auf einen angemeldeten Kurs werden die Details dieses Kurses angezeigt. Hier hat der registrierte Benutzer die Möglichkeit sich zu Kurseinheiten an-/abzumelden und sich aus dem Kurs auszutragen.
	\item \Func{Anzeigen der Profil - Seite.}
	Der registrierte Benutzer besitzt eine Profilseite auf welcher seine Daten gesammelt angezeigt werden. 
	\item \Func{Ändern der eigenen Benutzerdaten.}
	Der registrierte Benutzer kann durch Klicken der Schaltfläche 'Benutzerdaten bearbeiten' seine eingegebenen Daten bearbeiten. Sollte die E-Mail-Adresse geändert werden, wird an die neue E-Mail-Adresse wie beim Registrierungsprozess eine Bestätigungsmail mit einem neuen Authentifizierungslink geschickt, um die neue E-Mail-Adresse auf Korrektheit zu überprüfen.
	Der registrierte Nutzer kann alle seine Benutzerdaten ändern mit Ausnahme seines Kontostands und der vom System generierten Identifikationsnummer.
	\item \Func{Hochladen eines Profilbilds.}
	Der registrierte Benutzer kann ein Profilbild hochladen. Als Default ist ein Dummy - Bild gesetzt.
	\item \Func{Anzeigen des aktuellen Kontostands.}
	Der registrierte Benutzer kann den aktuellen Kontostand seines Guthabenkontos einsehen. 
	\item \Func{Aus dem System abmelden mittels 'Abmelden' - Schaltfläche.}
	Der registrierte Benutzer kann sich  von jeder Seite des Systems aus durch Klicken der Schaltfläche 'Abmelden' aus dem System abmelden und wird auf die Anmeldeseite weitergeleitet.
	\item \FuncW{Anzeige der nächsten Termine.}
	Die nächsten zehn Termine (Name, Datum/Zeit, Ort) des registrierte Benutzers werden gesammelt in Form einer Auflistung angezeigt.
\end{itemize}		
\subsection{Terminplaner}
\begin{itemize}
	\item \Func{Terminplaner.}
	Der registrierte Benutzer hat einen persönlichen Terminplaner. Dieser wird in der Wochenansicht dargestellt. Der Terminplaner unterstützt nur die Anzeige von einstündigen Slots. Ist ein Termin kürzer, wird dennoch der Slot für die ganze Stunde belegt. Ist der Termin länger als eine Stunde werden dementsprechend mehrere einstündige Slots belegt.
	Außerdem ist darauf zu achten, dass die Slots jeweils zur vollen Stunde beginnen und zur nächsten vollen Stunde enden. (Analog dem Prinzip eines Stundenplans)
	\item \Func{Anzeigen des Terminplaners.}
	Der persönliche Terminplaner des registrierten Benutzers kann angezeigt werden. 
	\item \Func{Eintragen von eigenen Terminen in den Terminplaner.} 
	Der registrierte Benutzer kann eigene Termine in seinen Terminplaner eintragen. Es können nur Termine eingetragen werden, wenn zur entsprechenden Zeit noch kein anderer Termin im Terminplaner vorhanden ist. 
	\item \Func{Automatische Eintragung von angemeldeten Kurseinheiten.}
	Die angemeldeten Kurseinheiten des registrierten Benutzers werden automatisch in dessen Terminplaner eingetragen.
	\item \Func{Automatische Entfernung von abgemeldeten Kurseinheiten.}	
	Meldet sich der registrierte Benutzer aus einer Kurseinheit ab oder fällt eine Kurseinheit aus, so wird diese aus dem Terminplaner gelöscht.
\end{itemize}   
\subsection{Kursfunktionen}

\begin{itemize}
	\item \Func{Detailansicht der Kurse.}
	Der registrierte Nutzer kann sich eine Detailansicht der angebotenen Kurse anzeigen lassen. Er erreicht diese indem er in der Kurssuche auf den gefundenen Kurs doppelklickt. In der Detailansicht kann sich der Nutzer, wenn er zum Kurs bereits angemeldet ist(\FuncBlue{/F20-150/}) zu Kurseinheiten anmelden(\FuncBlue{/F20-170/}) oder von Kurseinheiten abmelden(\FuncBlue{/F20-180/}).
	\item \Func{Zu Kurs anmelden.} 
	Der registrierte Nutzer kann sich zu beliebig vielen Kursen anmelden. 
	Dafür klickt er in der Detailansicht des gewünschten Kurses auf die Schaltfläche 'Anmelden'. Die Detailansicht des Kurses erreicht er indem er in der Kurssuche auf den gewünschten Kurs doppelklickt(\FuncBlue{/F20-140/}).
	\item \Func{Von Kurs abmelden.} 
	Der registrierte Nutzer kann sich von Kursen abmelden zu denen er angemeldet ist. Dafür klickt er in der Detailansicht des gewünschten Kurses auf die Schaltfläche 'Abmelden'. Die Detailansicht des Kurses erreicht er indem er in der Kurssuche auf den gewünschten Kurs doppelklickt(\FuncBlue{/F20-140/}). Meldet sich der registrierte Benutzer von einem Kurs ab, meldet er sich damit auch automatisch von allen Kurseinheiten ab(\FuncBlue{/F20-180/}).
	\item \Func{Zu Kurseinheit anmelden.}
	Der registrierte Benutzer kann sich zu beliebig vielen Kurseinheiten anmelden. Um sich zu Kurseinheiten anmelden zu können, muss der Nutzer bei diesem Kurs angemeldet sein(\FuncBlue{/F20-150/}).
	Um sich zu Kurseinheiten eines Kurses anmelden zu können, muss der Nutzer in die Detailansicht des Kurses wechseln. Dort findet er die zu diesem Kurs verfügbaren Einheiten und deren Status. Ist eine Kurseinheit noch nicht voll und ist der Nutzer noch nicht zu dieser Einheit angemeldet, kann er sich anmelden. Der Nutzer hat außerdem die Möglichkeit durch Klicken der Schaltfläche 'Alle Einheiten auswählen' und anschließender Betätigung der Schaltfläche 'Anmelden' sich zu allen Einheiten des Kurs auf einmal anzumelden. Vor Betätigen der Schaltfläche 'Anmelden' hat der Nutzer aber auch noch die Möglichkeit einzelne Einheiten von der Anmeldung auszuschließen. Voraussetzung um sich zu kostenpflichtigen Kurseinheiten anmelden zu können ist ein ausreichendes Guthaben auf dem Konto bzw. keine unzulässige Überziehung des Guthabenkontos.
	\item \Func{Von Kurseinheit abmelden.}
	Der registrierte Benutzer kann sich von Kurseinheiten abmelden, zu denen er angemeldet ist. Um sich von Kurseinheiten eines Kurses abmelden zu können, muss der Nutzer in die Detailansicht des Kurses wechseln. Dort findet er die zu diesem Kurs verfügbaren Einheiten.
	Er kann nun die Einheiten auswählen von denen er sich abmelden will und durch Betätigen der Schaltfläche 'Abmelden' meldet er sich von den Einheiten ab. Der für bezahlte kostenpflichtige Einheiten wird auf das Konto des registrierten Benutzers zurück gebucht(\FuncBlue{/F20-260/}).
	\item \Func{Kursangebot anzeigen - erweitert.}
	Der registrierte Benutzer hat die Möglichkeit sich das Kursangebot anzeigen zu lassen. Dafür wählt er die entsprechende Registerkarte aus.Als Default wird dabei das Angebot des aktuellen Tages angezeigt. Als weitere Anzeigezeiträume stehen 'Wochenangebot' und 'Gesamtes Angebot' zur Verfügung. Die Anzeigezeiträume werden über eine Drop - Down - Liste ausgewählt und durch Klicken der Schaltfläche 'Anzeigen' wird die Auswahl angezeigt(vgl. \FuncBlue{/F10-40/}).
	\item \Func{Nach Kursen suchen - erweitert.}
	Der registrierte Nutzer hat die Möglichkeit das gesamte Kursangebot zu durchsuchen. Über eine Drop-Down-Liste kann der Nutzer einen Suchparameter auswählen. Nach Eingabe eines Suchbegriffs und Klicken der Schaltfläche 'Suchen' wird das Kursangebot nach dem Suchbegriff bzgl. des Suchparameters durchsucht und das Ergebnis angezeigt. 
	Als Suchparameter sind vorgesehen 'KursId', 'Kursname', 'Ort', 'Trainer', 'Datum', 'Zeit', 'Kostenpflichtig', 'Kostenlos'.
	\item \Func{Kurse sortieren - erweitert.}
	Der registrierte Nutzer hat die Möglichkeit das angezeigte Kursangebot zu sortieren. Als Sortiermöglichkeiten sind vorgesehen alphabetisch nach 'Kursname', 'Ort' oder 'Trainer' und chronologisch nach 'Datum' oder 'Uhrzeit'.
	\item \Func{Für Kursbenachrichtigungen eintragen}
	Der registrierte Nutzer hat die Möglichkeit sich bei der Anmeldung zu einem Kurs durch Auswahl einer Checkbox  für Benachrichtigungen zu diesem Kurs zu registrieren oder nicht.
	
\end{itemize}


\subsection{Bezahlfunktionen}
\begin{itemize}
	\item \Func{Guthabenkonto aufladen - online}
	Der registrierte Nutzer besitzt ein Guthabenkonto. Dieses kann er online mittels Kreditkarte aufladen. Die Kreditkarten Abwicklung erfolgt dabei
	über die Infosun-Bank. Die Übertragung der Kreditkartendaten erfolgt mittels SSL. Der aufgeladene Betrag ist sofort auf dem Guthabenkonto verfügbar und kann auch sofort verwendet werden.
	\item \Func{Guthabenkonto aufladen - offline}
	Der registrierte Nutzer kann sein Guthabenkonto auch offline aufladen. Die ist mittels Überweisung oder Barzahlung an den Betreiber oder einen Trainer möglich. 
	Der Nutzer überweist dafür den Betrag unter Angabe seine BenutzerID und seines Namens  auf das Konto des Betreibers oder bezahlt bar. Der bezahlte Betrag ist erst auf dem Guthabenkonto des registrierten Benutzers vorhanden, wenn er vom Betreiber oder einem der Trainer gutgeschrieben wurde.
	\item \Func{Kurseinheiten bezahlen}
	Die Bezahlung von Kurseinheiten wird automatisch bei der Anmeldung zur Kurseinheit erledigt(\FuncBlue{/F20-170/}). Der registrierte Nutzer kann sich also auch nur zu der Kurseinheit anmelden, wenn sein Guthaben auf dem Konto für den Preis der Einheit ausreicht bzw. sein Kontostand nicht den vom Betreiber festgelegten Überziehungskredit überschreitet.
	\item \Func{Rückbuchung des Kurspreises bei Abmeldung oder Kursausfall}
	Meldet sich der registrierte Nutzer von einer Kurseinheit ab oder fällt die Kurseinheit aufgrund zu geringer Teilnehmerzahl oder wegen eines anderen Grunds aus, so wird der Kaufpreis der Einheit automatisch auf das Konto des registrierten Nutzers zurück gebucht.
\end{itemize}	

\section{Trainer}
Jedem Trainer stehen auch alle Funktionen eines registrierten Benutzers zur Verfügung. Darüber hinaus kann ein Trainer auf folgenden Funktionen zurückgreifen.
\stepHc
\subsection{Kurse}
\begin{itemize}
	\item \Func{Eigene Kurse anzeigen.}
	Auf der Trainerseite werden die Kurse, welche der Trainer trainiert gesammelt angezeigt.
	\item \Func{Eigenen Kurs editieren.}
	Der Trainer hat die Möglichkeit, die Kurse, die er  trainiert zu editieren. Dafür klickt er in seinen Trainingskursen den gewünschten Kurs an und wird auf die 'Kurs bearbeiten' - Seite weitergeleitet. 
	Dort kann er die Kursdaten bearbeiten. Er kann alle Kursdaten verändern mit Ausnahme der vom System generierten KursId. Durch Betätigen der 'Speichern' - Schaltfläche werden die Kursdaten gespeichert.
	\item \Func{Eigenen Kurs löschen.}
	Der Trainer hat die Möglichkeit, die Kurse, die er  trainiert zu löschen. Dafür klickt er in seinen Trainingskursen den gewünschten Kurs an und wird dann auf die 'Kurs bearbeiten' - Seite weitergeleitet. 
	Dort kann er den Kurs durch Betätigen der 'Kurs löschen' - Schaltfläche löschen.
	\item \Func{Kurseinheit anlegen.}
	Der Trainer hat die Möglichkeit Kurseinheiten für einen Kurs anzulegen, welchen er trainiert. Dafür wechselt er in die 'Kurs bearbeiten' - Oberfläche, trägt die Daten der Kurseinheit ein und legt die Kurseinheit durch Betätigen der Schaltfläche 'Kurs anlegen' an. Die Kurseinheiten können nur in dem Zeitraum zwischen Startzeitpunkt und Endzeitpunkt des zugehörigen Kurses stattfinden.
	\item \Func{Kurseinheit editieren.}
	Der Trainer hat die Möglichkeit Kurseinheiten der eigenen Kurse zu editieren. Dafür klickt er in der Detailansicht des Kurses die 'Bearbeiten' - Schaltfläche der Kurseinheit und wird auf die 'Kurseinheit bearbeiten' weitergeleitet. Hier kann der Trainer die Kurseinheitsdaten bearbeiten. Außerdem kann der Trainer hier
	manuell durch Eingabe der Daten eine Teilnehmer zur Kurseinheit hinzufügen.
	\item \Func{Kurseinheit löschen.}
	Der Trainer kann Kurseinheiten löschen. Es ist möglich durch Auswahl der entsprechenden CheckBoxes auf der 'Kurs bearbeiten' - Seite auch mehrere Kurse gleichzeitig zu löschen.
	\item \Func{Teilnehmer zu Kurs hinzufügen.}
	Der Trainer hat die Möglichkeit manuell durch Eingabe der Daten eines Benutzers diesen als Teilnehmer zu einer einzelnen Kurseinheit einzutragen. Dafür muss der Trainer auf die Bearbeiten - Seite der entsprechenden Kurseinheit wechseln.
	\item \Func{Teilnahmer aus Kurs entfernen.}
	Der Trainer kann einzelne Teilnehmer aus einem seiner eigenen Kurse entfernen. Dafür muss er in der 'Kurs bearbeiten' - Oberfläche die zu entfernenden Teilnehmer auswählen und kann diese durch Betätigen der 'Löschen' - Schaltfläche aus der Kurseinheit entfernen. Es ist damit auch möglich mehrere Teilnehmer gleichzeitig zu entfernen.
	\item \Func{Trainer zu Kurs hinzufügen.}
	Der dem Kurs zugeordnete Trainer kann dem Kurs noch weitere Trainer hinzufügen. Dafür wechselt zur 'Kurs bearbeiten' - Seite und trägt dort die Daten des zusätzlichen Trainers ein. Durch Betätigen der 'Hinzufügen' - Schaltfläche wird der Trainer hinzugefügt.
	\item \Func{Trainer aus Kurs entfernen.}
	Der dem Kurs zugeordnete Trainer kann dem Kurs noch weitere Trainer hinzufügen. Dafür wechselt er zur 'Kurs bearbeiten' - Seite und trägt dort die Daten des zusätzlichen Trainers ein. Durch Betätigen der 'Hinzufügen' - Schaltfläche wird der Trainer hinzugefügt.
	\item \FuncW{InvitedGuests zu Kurs hinzufügen.}
	Der Trainer hat die Möglichkeit nicht im System registrierte Benutzer zu einzelnen seiner Kurseinheiten einzutragen. Dabei spielt es keine Rolle, ob die Kurseinheit kostenpflichtig ist oder nicht.
\end{itemize}

\subsection{Benachrichtigungen}
\begin{itemize}
	\item \Func{Benachrichtigung an Benutzergruppe senden.}
	Der Trainer kann Benachrichtigungen an alle Teilnehmer seines Kurses oder nur an die Teilnehmer einer Kurseinheit seines Kurses senden.
	\item \Func{Benachrichtigung bei Kursänderung senden.}	
	Nimmt der Trainer Änderungen an einem seiner Kurse oder seiner Kurseinheiten vor, so kann er entscheiden, ob er alle Kursteilnehmer, nur die Teilnehmer der betreffenden Kurseinheit oder keinen Teilnehmer benachrichtigt.
\end{itemize}

\subsection{Verwaltung}
\begin{itemize}
	\item \Func{Account bestätigen}
	Der Trainer hat die Möglichkeit Accounts zu bestätigen, falls dies durch den Administrator in den Registrierungsbestätigung(\FuncBlue{/F40-140/}) so festgelegt wurde. Registriert sich ein Benutzer neu und ist die Bestätigung von Accounts auf manuell durch den Trainer eingestellt, so erhalten bei erfolgreicher Registrierung des Nutzers alle Trainer eine E-Mail mit dem Aktivierungslink für den Account. Sobald dieser Link von einem Trainer erstmalig geklickt wird, ist der Account freigeschaltet. Erst jetzt kann sich der registrierte Nutzer im System anmelden. Mehrmaliges Betätigen des Links hat keine Auswirkung.
\end{itemize}


\section{Systemadministratoren}
Jedem Systemadministrator stehen auch alle Funktionen eines Trainers zur Verfügung. Darüber hinaus kann ein Systemadministrator auf folgende Funktionen zurückgreifen.
\stepHc
\subsection{Kurse}
\begin{itemize}
	\item \Func{Kurs anlegen.}
	Der Systemadministrator kann einen neuen Kurs anlegen. Dafür betätigt der Administrator die entsprechende Schaltfläche auf der Administratorseite und wir auf die 'Kurs anlegen' - Seite weitergeleitet. Die Anwendung generiert automatisch eine für den Kurs im System eindeutige Identifikationsnummer. Der Administrator hat die Möglichkeit noch weitere Daten, wie eine Beschreibung zum Kurs einzugeben. Notwendige Eingaben sind der Startzeitpunkt, der Endzeitpunkt und mindestens ein Trainer für den Kurs. Innerhalb des Zeitraum von Startzeitpunkt bis Endzeitpunkt können Kurseinheiten stattfinden(\FuncBlue{/F30-40/}).
	\item \Func{Kurse verwalten.}
	Der Systemadministrator besitzt eine Übersicht über alle Kurse. Diese erreicht er über die Schaltfläche 'Kurse verwalten' der Administratorseite. Auf dieser Seite wird als Default das gesamte Kursangebot angezeigt. Der Administrator hat aber die Möglichkeit die Anzeige nach Parametern einzuschränken. Dafür wählt er aus einer Drop - Down - Liste einen Parameter und gibt in ein Eingabefeld seinen Suchbegriff ein. Durch Betätigen der Schaltfläche 'Anzeigen' werden die gesuchten Kurse angezeigt. Durch Klicken auf ein einen Kurs wird der Administrator auf die 'Kurs bearbeiten' - Seite weitergeleitet.
	\item \Func{Kurs bearbeiten.}
	Der Systemadministrator hat die Möglichkeit einen beliebigen Kurs zu bearbeiten. Dafür sucht er den Kurs über die 'Kurse verwalten' - Seite und wechselt durch Doppelklicken auf den gesuchten Kurs in die 'Kurs bearbeiten' - Oberfläche. Der Administrator kann nun die Kursdaten editieren mit Ausnahme der vom System generierten KursId. Des Weiteren kann er Kurseinheiten anlegen, löschen oder editieren, Trainer zum Kurs hinzufügen oder entfernen und die Teilnahmen des Kurses editieren.
	\item \Func{Kurs löschen.}
	Der Systemadministrator hat die Möglichkeit einen beliebigen Kurs zu löschen. Dafür sucht er den zu löschenden Kurs über 'Kurse verwalten' - Seite und wechselt durch Doppelklicken auf den gesuchten Kurs in die 'Kurs bearbeiten' - Oberfläche. Durch Betätigen der 'Kurs löschen' - Schaltfläche wird der Kurs gelöscht.
\end{itemize}

\subsection{Benutzer}
\begin{itemize}
	\item \Func{Neuen Benutzer anlegen.}
	Der Systemadministrator hat die Möglichkeit einen neuen Benutzer anzulegen. Dafür betätigt er die Schaltfläche 'Benutzer anlegen' auf der Administratorseite und wird auf eine Seite 'Neuen Benutzer anlegen' weitergeleitet. 
	Die Anwendung generiert automatisch eine im System eindeutige Identifikationsnummer. Der Systemadministrator kann die Benutzerdaten eingeben, ein Initialpasswort setzen und die Rolle des Benutzers festlegen. Die Benutzerrolle wird mittels einer Drop-Down - Liste aus den Möglichkeiten 'Trainer', 'Administrator','Benutzer' ausgewählt. Durch Betätigen der Schaltfläche 'Speichern' wird der neue Nutzer angelegt.
	\item \Func{Benutzer verwalten.}
	Der Systemadministrator besitzt eine Übersicht über alle Benutzer. Diese erreicht er über die Schaltfläche 'Benutzer verwalten' der Administratorseite. Auf dieser Seite werden als Default alle Benutzer angezeigt. Der Administrator hat aber die Möglichkeit die Anzeige nach Parametern einzuschränken. Dafür wählt er aus einer Drop - Down - Liste einen Parameter aus und gibt in ein Eingabefeld seinen Suchbegriff ein. Durch Betätigen der Schaltfläche 'Anzeigen' werden die gesuchten Benutzer angezeigt. Durch Klicken auf ein einen Benutzer wird der Administrator auf die 'Benutzer bearbeiten' - Seite weitergeleitet.
	\item \Func{Benutzer bearbeiten.}
	Der Systemadministrator hat die Möglichkeit einen beliebigen registrierten Nutzer zu bearbeiten. Dafür sucht er den Benutzer über die 'Benutzer verwalten' - Seite und wechselt durch Doppelklicken auf den gesuchten Benutzer in die 'Benutzer bearbeiten' - Oberfläche. Der Administrator kann nun die Benutzerdaten editieren oder die Benutzerrolle verändern. Der Administrator kann alle Benutzerdaten verändern mit Ausnahme der vom System generierten Identifikationsnummer. Durch Betätigen der 'Speichern' - Schaltfläche werden die Daten gespeichert.
	\item \Func{Benutzer löschen.}
	Der Systemadministrator hat die Möglichkeit einen beliebigen Benutzer zu löschen. Dafür sucht er den zu löschenden Benutzer über die 'Benutzer verwalten' - Seite und wechselt durch Doppelklicken auf den gesuchten Benutzer in die 'Benutzer bearbeiten' - Oberfläche. Durch Betätigen der 'Benutzer löschen' - Schaltfläche wird der Benutzer gelöscht. Nur der Administrator hat die Möglichkeit Benutzer zu löschen. Andernfalls könnte ein \glqq Nicht - Administrator\grqq \ einen Benutzer und damit auch sein Guthaben bzw. seine eventuellen Schulden löschen.
	\item \Func{Benutzerrolle aufwerten.}
	Der Systemadministrator hat die Möglichkeit die Rolle jedes Nutzers aufzuwerten. Als Benutzerrollen sind 'Benutzer', 'Trainer' und 'Administrator' vorgesehen. Die Benutzerrolle kann auf der 'Benutzer bearbeiten' - Seite aufgewertet werden.
	\item \Func{Benutzerrolle abwerten.}
	Der Systemadministrator hat die Möglichkeit die Rolle jedes Nutzers abzuwerten. Als Benutzerrollen sind 'Benutzer', 'Trainer' und 'Administrator' vorgesehen. Die Benutzerrolle kann auf der 'Benutzer bearbeiten' - Seite abgewertet werden.
	\item \Func{Benutzer suchen.}
	Der Systemadministrator hat die Möglichkeit nach Benutzern zu suchen. Dafür wählt er in der 'Kurse verwalten' - Oberfläche die entsprechenden Suchparameter und gibt den Suchbegriff ein. Durch Betätigen der Schaltfläche 'Anzeigen' wird das Suchergebnis angezeigt.
\end{itemize}

\subsection{Verwaltung}
\begin{itemize}
	\item \Func{Statistiken anzeigen}
	Der Systemadministrator hat die Möglichkeit sich Statistiken bezüglich der Einnahmen anzusehen. Er kann dafür aus verschiedenen Parametern wählen, welche Statistik angezeigt werden soll. Die Auswahl der Parameter erfolgt über eine Drop-Down - Liste. Vorgesehene Parameter sind für die Statistik sind Einnahmen pro 'Tag', pro 'Woche', pro 'Trainer', pro 'Kurs'. Bei Wahl eines der Parameter 'Tag', 'Woche' oder 'Trainer' ist ein graphische Darstellung in Form eines Säulendiagramms vorgesehen. Bei Parameter 'Tag' werden die Einnahmen der letzten sieben Tage angezeigt, bei Parameter 'Woche' die Einnahmen der letzten vier Wochen. Bei Wahl des Parameters 'Kurs' ist aufgrund der Menge von Kursen eine tabellarische Anzeige vorgesehen. Durch Betätigen der Schaltfläche 'Anzeigen' wird die gewählte Statistik angezeigt.
	\item \Func{Guthabenkonto aufladen}
	Der Systemadministrator hat die Möglichkeit das Guthabenkonto von  registrierten Nutzern manuell aufzuladen. Damit kann er \glqq Offline - Aufladungen\grqq (vgl. \FuncBlue{/F20-240/}) des Guthabenkontos eines registrierten Nutzers auf dessen Konto Konto im System weitergeben. Die Kontoaufladung erreicht der Administrator über die entsprechende Schaltfläche auf der Administratorseite. Durch Betätigen der Schaltfläche 'Konto aufladen' wird er auf die Aufladeseite weitergeleitet. Dort kann der gibt der Administrator die Benutzerdaten(u.a. die BenutzerID) und den aufzuladenden Betrag ein. Durch Betätigen der Schaltfläche 'Speichern' wird das Konto des Nutzers aufgeladen.
	\item \Func{Überziehungskredit festlegen.}
	Der Systemadministrator kann festlegen, ob und gegebenenfalls um welchen Betrag ein registrierter Benutzer sein Guthabenkonto überziehen darf. Der Überziehungskredit nur global für alle Nutzer festgelegt werden. Dafür trägt der Systemadministrator auf der Administratorseite den Betrag, der maximal überzogen werden darf in das dafür vorgesehene Eingabefeld ein und bestätigt die Eingabe durch Betätigen der 'Speichern'-Schaltfläche.
	Falls kein Überziehungskredit gewährt werden soll, wird als Betrag einfach 0,00 eingegeben. Dieser Wert ist auch per Default gesetzt.
\end{itemize}
\subsection{Systemanpassung}
\begin{itemize}
	\item \Func{Logo hochladen.}
	Das Logo kann vom Systemadministrator editiert werden. Durch Klicken der 'Durchsuchen' - Schaltfläche gibt der Administrator den Pfad einer .jpg - Datei an, welche durch Betätigen der 'Speichern' - Schaltfläche hochgeladen wird. Bei Neustart der Anwendung wird die neue .jpg - Datei automatisch geladen und somit das neue Logo verwendet.
	\item \Func{Registrierungsbestätigung festlegen.}
	Die Art der Bestätigung einer Registrierung eines neuen Benutzers kann vom Systemadministrator festgelegt werden.
	Vorgesehen sind die E-Mail-Verifikation oder die manuelle Bestätigung durch den Administrator. Um die Verifikationsart festzulegen, wählt der Administrator die gewünschte Möglichkeit aus der entsprechenden Drop-Down - Liste und bestätigt die Auswahl durch Betätigen der 'Speichern'- Schaltfläche.
	\item \Func{Oberfläche editieren.}
	Die Oberfläche kann vom Systemadministrator editiert werden. Durch Klicken der 'Durchsuchen' - Schaltfläche gibt der Administrator den Pfad einer .CSS - Datei an, welche durch Betätigen der 'Speichern' - Schaltfläche hochgeladen wird. Bei Neustart der Anwendung wird die neue .CSS - Datei automatisch geladen und somit das bearbeitete Oberfläche verwendet.
	\item \Func{Impressum editieren.}
	Das Impressum kann vom Systemadministrator editiert werden. Durch Klicken der 'Durchsuchen' - Schaltfläche gibt der Administrator den Pfad einer .txt - Datei an, welche durch Betätigen der 'Speichern' - Schaltfläche hochgeladen wird. Bei Neustart der Anwendung wird die neue .txt - Datei automatisch geladen und somit das bearbeitete Impressum verwendet.
	\item \Func{Allgemeine Geschäftsbedingungen editieren.}
	Die allgemeinen Geschäftsbedingungen können vom Systemadministrator editiert werden. Durch Klicken der 'Durchsuchen' - Schaltfläche gibt der Administrator den Pfad einer .txt - Datei an, welche durch Betätigen der 'Speichern' - Schaltfläche hochgeladen wird. Bei Neustart der Anwendung wird die neue .txt - Datei automatisch geladen und somit die bearbeiteten allgemeinen Geschäftsbedingungen verwendet.
\end{itemize}
		
		

\chapter{Produktdaten}
 \label{Produktdaten}
        
	
    \section{System}
	    
	    
	    
	    \subsection{SMTP-Server}
	    
	    
	    
	    	
	\section{Kurs}
	   
	    
	    
    
    
    		
    \section{Registrierte Benutzer}
	  
	    
	    
	   
	    
    
\chapter{Produktleistungen}
	

	\section{Produktleistungen auf Serverseite}
		
	\section{Produktleistungen auf Clientseite}
	
		
		
 
\chapter{Benutzungsoberfläche}
    
    
    \subsection{Anonymer Benutzer}
       	
       	
       	
    \subsection{Registrierter Benutzer}
       
       	
       	    
    \subsection{Kursadministrator}
        
        

    \subsection{Systemadministrator}
        
        
            
   
    
    \section{Verwaltungsoberfläche des Kursadministrators}
    
   
       
    
    \section{Kurseigenschaften}
    
    
       
    
    
    \section{Navigationsdiagramme}
        \subsection{Anonymer Benutzer}
            
            
            
        \subsection{Registrierter Benutzer}
             
            
            
        \subsection{Kursadministrator}
            
             
            
        \subsection{Systemadministrator}
           
            
            
            
         
\chapter{Qualitätsbestimmungen}

\begin{table}[h]
 
    \begin{center}
    \begin{tabular}{|l|c|}
    \hline 
    \rule[-1ex]{0pt}{2.5ex} \textbf{Qualitätskriterium} & \textbf{Bedeutung} \\ 
    \hline 
    \rule[-1ex]{0pt}{2.5ex} Funktionalität &  \\ 
    \hline 
    \rule[-1ex]{0pt}{2.5ex} Zuverlässigkeit &  \\ 
    \hline 
    \rule[-1ex]{0pt}{2.5ex} Benutzbarkeit &  \\ 
    \hline 
    \rule[-1ex]{0pt}{2.5ex} Effizienz &  \\ 
    \hline 
    \rule[-1ex]{0pt}{2.5ex} Änderbarkeit &  \\ 
    \hline 
    \rule[-1ex]{0pt}{2.5ex} Übertragbarkeit &  \\ 
    \hline   
    \end{tabular}  
    \end{center}
    \caption{+: weniger wichtig, ++: wichtig, +++: sehr wichtig} 
    \label{qTabelle}   
\end{table}
    
Darüber hinaus sollten noch folgende Qualitätsmerkmale genannt werden, die in der Norm nicht berücksichtigt werden:
 
\resetAllCounter
\newcommand{\Test}[1]{\stepcounter{Lc}\textcolor{Brown}{\textbf{/T\arabic{Hc}0-\arabic{Lc}0/} #1} \\}
\newcommand{\RefFuncBlue}[1]{\textcolor{Blue}{\textbf{#1}}}
\newcommand{\RefFuncGreen}[1]{\textcolor{Green}{\textbf{#1}}}
\chapter{Globale Testszenarien und Testfälle}
 

	\section{Testfälle für den Systemadministrator ohne bestehenden Datensatz}
		\subsection{Setup}
			\begin{itemize}
				\item \Test{Website einrichten} 
				Der Administrator ... (vgl. \RefFuncBlue{/F10-20/})		
				\item \Test{Titel der Seite wählen} 
				Der Administrator ... 		
			\end{itemize}			
		\subsection{Erstellung und Verwaltung}
			\begin{itemize}
				\item \Test{Benutzer erstellen} 
				Der Administrator ... 		
				\item \Test{Impressum eingeben} 
				Der Administrator ... 		
			\end{itemize}						
		\subsection{Systemdarstellung}
			
			
			
	\section{Testfälle für den Kursadministrator}
		\stepHc
		\begin{itemize}
			\item \Test{Kurs einrichten} 
			Der Kursleiter ... 		
			\item \Test{Titel der Seite wählen} 
			Der Kursleiter ... 		
		\end{itemize}	
		
			
	\section{Testfälle für anonymer Benutzer}
			\stepHc
			\begin{itemize}
				\item \Test{Kursangebot ansehen} 
				Der anonyme Benutzer klickt auf der Startseite auf den Button 'Kursangebot'	und gelangt auf die Seite mit den Kursangeboten. Angezeigt wird zunächst das aktuelle Tagesangebot.	
				\item \Test{Registrierung} 
				Über die Schaltfläche 'Login' gelangt der anonyme Benutzer auf die Loginseite und von hier aus über den Button 'registrieren' auf die nächste Seite mit dem Registrierungsformular. 		
				\begin{itemize}
					\item Der Benutzer gibt bei der Registrierung ein zu kurzes Passwort ein und erhält nach Drücken des 'registrieren'-Buttons eine Fehlermeldung.	
					\item Der Benutzer gibt zwei verschiedene Passwörter ein und erhält nach Drücken des 'registrieren'-Buttons eine Fehlermeldung.
					\item Der Benutzer hat seine Mailadresse nicht angegeben und erhält nach Drücken des 'registrieren'-Buttons eine Fehlermeldung.	
					\item Der Benutzer meldet sich mit folgenden Daten an: 
						\begin{itemize}
							\item Benutzername: 'Kathi5'
							\item Passwort: 'bSdFg7HjK'
							\item Passwort: 'bSdFg7HjK'
							\item E-Mailadresse: 'katharina\_hoelzl@web.de'
							\item Anrede: 'Frau'
							\item Name: 'Hölzl'
							\item Vorname: 'Katharina'
							\item Geburtsdatum: '29.05.1993'
							\item Adresse: 'Am Kastenfeld 39'
							\item Postleitzahl: '94081'
							\item Ort: 'Fürstenzell'
						\end{itemize}
					Der Benutzer wählt außerdem die Option 'E-Mail-Verifizierung' aus der Liste der Verifizierungsarten aus. Nach Drücken des 'registrieren'-Buttons erhält der Benutzer eine Verifizierungsmail an die angegebene Mailadresse. Diese enthält den Aktivierungslink. Durch diesen wird das Konto aktiviert und der Benutzer auf die Loginseite weitergeleitet.
				\end{itemize}
			\end{itemize}	
		
	\section{Testfälle für den registrierten Benutzer}
		\stepHc
		\begin{itemize}
			\item \Test{Einloggen ins System} 
			Der Benutzer 'Kathi5' klickt auf der Startseite	auf den 'Login'-Button und wird auf die Loginseite weitergeleitet.
			\begin{itemize}
				\item Der Benutzer gibt seinen Benutzernamen 'Kathi5' und das falsches Passwort 'wertz6uio' ein. Nach Drücken des 'Login'-Buttons erhält er die Fehlermeldung 'Ihr Benutzername oder Ihr Passwort ist falsch'.
				\item Der Benutzer gibt den falschen Benutzernamen 'Kathi1' und das richtige Passwort 'bSdFg7HjK' ein. Nach Drücken des 'Login'-Buttons erhält er die Fehlermeldung 'Ihr Benutzername oder Ihr Passwort ist falsch'.
				\item Der Benutzer gibt seinen Benutzernamen 'Kathi5' und kein Passwort ein. Nach Drücken des 'Login'-Buttons erhält er die Fehlermeldung 'Ihr Benutzername oder Ihr Passwort ist falsch'.
				\item Der Benutzer gibt seinen Benutzernamen 'Kathi5' und das richtige Passwort 'bSdFg7HjK' ein. Nach Drücken des 'Login'-Buttons wird der Benutzer auf die Seite mit seinen Kursen weitergeleitet.
			\end{itemize}
			\item \Test{Konto aufladen} 
			Der Benutzer 'Kathi5' klickt in seinem Profil auf die Schaltfläche 'Konto aufladen' und wird auf die Seite 'Kontoaufladung' weitergeleitet. Hier gibt Kathi5 folgende Daten ein:
				 \begin{itemize}
				 	\item Benutzerid: ???
				 	\item Name: 'Hölzl'
				 	\item Vorname: 'Katharina'
				 	\item Kreditinstitut: ???
				 	\item Kreditkartennummer: ???
				 	\item Betrag: 50 
				 \end{itemize}	
			Anschließend klickt der Nutzer auf den Button 'Aufladen' und erhält eine Erfolgsmeldung.
			\end{itemize}
			
			\begin{itemize}
				\item \Test{Kurs anmelden} 
				Der Benutzer 'Kathi5' klickt im Register 'Meine Kurse' auf die Schaltfläche 'Weitere Kurse' und bekommt die einzelnen Kurse auf der Kursangebotsseite angezeigt. Im Kursangebot sucht 'Kathi5' nach dem Kurs 'Yoga'. Durch Doppelklicken auf den gefundenen Kurs erhält der Benutzer die Detailansicht des Kurses 'Yoga', welcher regelmäßig am Dienstag von 16 Uhr bis 17:30 Uhr angeboten wird und kostenlos ist. 'Kathi5' meldet sich durch klicken auf den Button 'Anmelden' in der Detailansicht für den Kurs 'Yoga' an.	
				
				\item \Test{Kurseinheiten anmelden} 
				Der Nutzer 'Kathi5'	klickt im Register 'Meine Kurse' auf den Kurs 'Yoga' und erhält dessen Detailansicht. Durch Klicken des Buttons 'Alle Einheiten auswählen' und anschließend der Schaltfläche 'Anmelden' trägt sich der Benutzer nun in alle verfügbaren Kurseinheiten ein.
				
				\item \Test{Kurs bezahlen}
				Der Benutzer 'Kathi5' meldet sich außerdem auf der Kursangebotsseite für den Kurs 'Standardtanz' an. Durch Doppelklick auf den Kurs erhält der Nutzer die Detailansicht und wählt die folgenden Einheiten aus:
					\begin{itemize}
						\item Montag 04.05.2015 18-20 Uhr
						\item Montag 11.05.2015 18-20 Uhr
						\item Montag 25.05.2015 18-20 Uhr		
					\end{itemize}	
				Durch Drücken der Schaltfläche 'Anmelden' trägt sich 'Kathi5' in die ausgewählten Kurseinheiten ein. Da eine Kurseinheit 15 Euro kostet werden automatisch durch das Anmelden 45 Euro von seinem Guthabenkonto abgebucht.
				
				\item \Test{eigenen Termin in den persönlichen Terminplaner eintragen}
				Über das Register 'Terminplaner' lässt sich 'Kathi5' ihren persönlichen Terminplaner anzeigen. Es existieren im Planer bereits der Yogakurs am Dienstag von 16 Uhr bis 17:30 Uhr und der Kurs 'Standardtanz' am Montag von 18 bis 20 Uhr. Der Benutzer trägt unter dem Terminplaner in der Zeile 'neuer Kurs' folgende Daten ein:
					\begin{itemize}
						\item Name: Fußball
						\item Datum: 14.05.2015
						\item Zeit von 17:00 bis 20:00 Uhr
						\item Ort: Trainingsfeld		
					\end{itemize}
				Durch Klicken auf die Schaltfläche 'Eintragen' wird dieser Termin in den persönlichen Terminplaner eingetragen.
				
				\item \Test{Kurseinheit abmelden und Rückbuchung des Kurspreises}
				Der Benutzer 'Kathi5' klickt in der Detailansicht des Kurses 'Standardtanz' auf die Kurseinheit 'Montag 25.05.2015 18-20 Uhr' und Drückt den Button 'Abmelden'. Dadurch wurde 'Kathi5' aus dieser Kurseinheit ausgetragen und der Betrag von 15 Euro wird automatisch auf das Guthabenkonto rückgebucht. 
				
				\item \Test{aktuelles Guthaben ansehen}
				Der Nutzer 'Kathi5' klickt auf die Registerkarte 'Mein Profil'. Hier wird der aktuelle Kontostand von 20 Euro angezeigt.
				
				\item \Test{Kursdetails empfangen}
				'Kathi5' klickt in der Detailansicht des Kurses 'Yoga' auf die das Auswahlfeld 'Kursdetails empfangen' und erhält so bei Veränderungen eine Infomail.
			\end{itemize}
\section{Testfälle mit Datensatz}






\chapter{Entwicklungsumgebung}
    \section{Software}
        \subsection{Betriebssysteme}
            
        \subsection{Dokumentation}
            
        \subsection{Datenbank}
           
        \subsection{Webserver}
            
        \subsection{Entwicklung}
           
        \subsection{sonstige Software}
           
    \section{Orgware}
       
    \section{Hardware}
        
\printglossaries
\end{document}
