\documentclass[a4paper]{scrreprt}
\usepackage[german]{babel}
\usepackage[german]{translator}
\usepackage[utf8]{inputenc}
\usepackage[T1]{fontenc}
\usepackage{ae}
\usepackage[bookmarks,bookmarksnumbered]{hyperref}
\usepackage{graphicx}
\usepackage{color}
\usepackage[dvipsnames]{xcolor}

%Glossar erzeugen
\usepackage[nonumberlist, toc, section=chapter, numberedsection=nolabel]{glossaries}
\makeglossaries

%Glossareinträge
\newglossaryentry{bezeichner}{name=Name, description=Beschreibung}


\begin{document}
	\thispagestyle{plain}

\begin{titlepage}
    \begin{center}

    	\begin{title}
        	\title{\Huge{\textbf{SEP 2015 - Team 3 \\ Pflichtenheft\\}}}

		\end{title}
		\hspace{3cm}

        	Software Engineering Praktikum \\
        	Sommersemester 2015\\
        	Universität Passau\\


        	Betreuer: Andreas Stahlbauer \\
        	\hspace{1,5cm}\\
        	Version: 1.0 \\
        	\hspace{1,5cm}\\
        	Datum: 01.04.2015\\[50pt]
        	Team 3 \\
    
        
        \begin{tabular}{ | l | l | l | l |}
            \hline
            \textbf{Matrikelnummer} & \textbf{Name} & \textbf{Phase} & \textbf{E-Mail}  \\ \hline
            63097 & Katharina Hölzl & Pflichtenheft & ... \\ \hline
            XXXXX & ... & Entwurf & ...  \\ \hline
            XXXXX & ... & Feinspezifikation  & ... \\ \hline
            XXXXX & ... & Implementierung  &  ... \\ \hline
            XXXXX & ... & Validierung & ... \\ \hline  
            XXXXX & ... &  Präsentation & ... \\ \hline
        \end{tabular}
    \end{center}
\end{titlepage}
 
 

% Inhaltsverzeichniss
\tableofcontents
 
% Glossar
\printglossaries
\end{document}
